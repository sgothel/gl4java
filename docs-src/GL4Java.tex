\documentclass{report} % Must use LaTex 2e
\usepackage{html,a4}
\latex{\usepackage{german}}

\begin{document}

\renewcommand{\thefootnote}{\alph{footnote}}
\renewcommand{\thempfootnote}{\alph{mpfootnote}}

\originalTeX

\pagestyle{headings}

\begin{titlepage}
  {\LARGE
      \begin{tabular}{p{20mm}l}
              & \htmladdnormallink{GL4Java}{http://www.jausoft.com}\\[3mm]
	      & Implementation Of A Native OpenGL-Interface \\
	      & to Java, X-Window and Windows (95/NT)\\[3mm]
	      & \emph{Version 2.4.0 Release 0}\\[5mm]
	      & \htmladdnormallink{Sven Goethel}{mailto:gl4java@jausoft.com} \\[25mm] 
	      & Jausoft - Sven Goethel Software development \\[25mm]
%	      \hline \\[20mm]
	      & 29. December 1997 (Diploma Thesis Closing)\\[50mm]
	      & 26th June 2000 (Last Changes)
      \end{tabular}
      }
\end{titlepage}

May be you want to check the follwing news about GL4Java directly:
\begin{itemize}
	\item The GL4Java top level homepage at \htmladdnormallink{http://www.jausoft.com}{http://www.jausoft.com}.
	\item The JavaDoc package-documentation at \htmladdnormallink{packages.html}{packages.html}.
	\item \emph{Download} GL4Java at \ref{download}, page \pageref{download}.
	\item The \emph{versions} file at \ref{versions}, page \pageref{versions}.
	\item The \emph{changes} file at \ref{changes}, page \pageref{changes}.
	\item The \emph{license} of GL4Java at \ref{license}, page \pageref{license}.
	\item The \emph{thanxs} file at \ref{thanxs}, page \pageref{thanxs}.
\end{itemize}


\tableofcontents

\chapter{Introduction}

\section{GL4Java Homepage}
Since GL4Java Version 1.0.2 the GL4Java-Homepage has its new top-level homepage.

\htmladdnormallink{http://www.jausoft.com}{http://www.jausoft.com} contains the top level homepage about GL4Java. Some links to other users are located here.

\section{About GL4Java}

Sorry that this English is not very good ;-) But I guess it's
better to write little english for everybody than to write German for
only a few.
\htmladdnormallink{Atilla Kolac}{mailto:a_kolac@bielefeld.netsurf.de}
and I
are working on the implementation of GL4Java\cite{GL4JavaHome}
for our diploma thesis under the supervision 
of Prof. Dr. Wolfgang Bunse\cite{ProfBunse}.

Now, i do still maintain GL4Java\cite{GL4JavaHome}.

The purpose of GL4Java is to use Java as a platform independed
programming language for OpenGL applications.

The development of GL4Java may be continued by us or other persons, 
if there are interesting users.

We would be very lucky, if we get some response, critic, inspirations,
bugfixes and participation. 

\section{About this Document}

This document was layouted with \LaTeX\ on Linux.
We used english as the main language, because this documentation is
published on the WWW-Server of the FH-Bielefeld\cite{GL4JavaHome2}. To convert this \LaTeX\ documentation we used latex2html\cite{Dragos96}.
The appendix includes the introduction about OpenGL and Java in the german language. We used german for these parts, because there are many documentations in the english language for these purposes \cite{Woo97} \cite{Fosner97} \cite{JavaTuto}. This appendix is not included in the html files.

The paperware we have to produce for the FH-Bielefeld includes a printout of
the javadoc-html pages of the base GL4Java classes.
We include a link to the javadoc pages in chapter \ref{GL4JavaHtml}, page \pageref{GL4JavaHtml}.



\section{Who wrote which part?}
The diploma thesis is splitten into two diploma theses:
\begin{itemize}
	\item Sven Goethel: Implementation Of A Native OpenGL-Interface to Java and X-Window
	\item Atilla Kolac: Implementation Of A Native OpenGL-Interface to Java and Windows-NT
\end{itemize}
Because we both worked together in design, implementation and documentation,
we publish one documentation and one package.

To help others to distinguish who of us is responsible for which part,
we marked Sven Goethel's tasks with a \dag, we marked Atilla Kolac's part with a \ddag, and we marked the parts which are written by both of us with nothing at all.
If the chapter is marked, the whole chapter was written by this person.
If only some sections are marked, only the specific section was written by this person.
If nothing is marked both persons wrote those chapter.

Since Version 1.1.0 Sven Goethel maintains this package.

\section{The Procedure\dag}
\subsection{History}
The OpenGL interface for Java work was started by Leo Chan\cite{Chan97}.
He implements the library with Java 1.0.2 native calls and with an extra window for OpenGL-Rendering.

Leo Chan's works was continued by Adam King\cite{King97}.
His OpenGL4Java is able
to be compiled with Java 1.1 and the OpenGL-Rendering is done in the calling
Java-Frame. He still uses the Java 1.0.2 native calls.

Tommy Reilly\cite{Reilly97}
participated to Adam King's work and the project's title changed to Jogl.
Jogl's big points lies in it's powerfull autoconfig and in it's
improved X-Window System functions - so mostly all Unices are supported.
Another point is the Win32 support. Sources and a precompiled dll are
distributed.
Jogl still differs from the true OpenGL naming convention.

Because of newly communication results, we are thinking about
joining the Jogl project. This does not means, that there will be no more
GL4Java ! We still support the GL4Java, with the OpenGL like API (naming
conventions) !

\subsection{Our Work}
We started developing GL4Java with Adam King's\cite{King97} OpenGL4Java.
As a matter of asynchronise development, 
the changes from Java 1.0.2 to JNI 1.1 native calls 
were made within the Jogl and GL4Java project parallel
- because we did not know of each others development.

Actual results and later changes in Jogl are ported and will be ported to GL4Java.
Like Jogl, GL4Java uses the Java 1.1 Java-Native-Interface (JNI).
Many incompatible changes to OpenGL4Java/Jogl were made, see the chapter \ref{changes} at page \pageref{changes}.

One big point of GL4Java is the used OpenGL naming convention,
support for glu* and glut* functions (glut support in the near future).

GL4Java extends OpenGL's API with an own naming convention.
Specialised known windowing functions, like glXSwapBuffers, have the prefix
glj, like gljSwap. 

Also the GLFrame class adds itself as a ComponentListener,
so we have a event-handler in java, like reshape for glut !

If we uses a own created Color-Window as the GL-Window
(not usefull for AIX, LINUX AND SOLARIS YET),
gljResize will be called if the componentResized is called (ComponentListener)
to resize the own created Window !

\subsection{Status}
We can announce the following stats about GL4Java Version 2.1.0.0
\begin{itemize}
	\item Stable versions with precompiled librarys for Linux, SunOS (Solaris), Windows (95/NT) and Macintosh.
	\item Runs on Java 1.1.5 and higher, tested on:
	\begin{itemize}
              	\item JDK 1.1.7 (win32 (plus jit), linux (plus jit (tya v3.0)))
		\item JDK 1.2 (Win32, Linux (Pre-V1,native,no-jit), Macintosh, ...)
		\item Java2 (Java1.2) Plug-In on Netscape 4.5 (Win32)
		\item Netscape 4.5 (Win32/Unix)
		\item MS-JVM (build 3186) Win32
		\item InternetExplorer4.0 with MS-JVM (build 3186) Win32
	\end{itemize}
	\item AIX stable version with some multithreading difficulties
	\item Many demonstrations of GL4Java, which also uses many Java GUI
	features.
	\item \htmladdnormallink{A JavaDoc HTML-Documentation}{packages.html}
\end{itemize}

\subsection{Targets}
If you use Mesa as your OpenGL driver, 
you should use Mesa Version 3.0 or higher !

The actual implementation of GL4Java supports the following UNICES

\begin{itemize}
	\item AIX (RS/6000) 4.2 with $\geq$ JDK 1.1.5 (without JIT\_COMPILER option)
	\item Linux (x86) 2.X with Mesa-3.0 and Metro-Link's GL and Xi's Accelerated OpenGL + $\geq$ JDK 1.1.7
	\item SunOS (Sparc) 5.X with Mesa-2.4 GL with $\geq$ JDK 1.1.5 (green threads)
	\item SGI-Irix 
	\item Macintosh 
\end{itemize}

but if you know about programming, i guess it only makes little afford to
support your unix.

And the following Windows
\begin{itemize}
	\item Windows NT 4.0 \& 95/98 (x86) with MS OpenGL-Library + $\geq$ JDK 1.1.7
	\item Windows NT 4.0 \& 95/98 (x86) with MS OpenGL-Library + $\geq$ MS-JVM (build 3186)
\end{itemize}

\chapter{Installation\dag}
\label{download}

\section{Obtain GL4Java}
You can download GL4Java at \htmladdnormallink{http://www.jausoft.com/Files/Java/1.1.X/GL4Java}{http://www.jausoft.com/Files/Java/1.1.X/GL4Java}, where you may can find older versions and new demos either.

Then go to this chapter and click on the following list,
or just download the appropiate file.

Please use some \emph{Save Link to ...} function of your browser !

It could happen that your browser unzip the package automatically,
then 'gunzip' or 'tar xzf' does not work of course - just use 'tar xf' then !

Windows users can use the WinZip utility to extract the tar-gzipped files,
but be sure to NOT convert LF/CR characters, because these files are binary !

\begin{itemize}

	\item \htmladdnormallink{The precompiled binaries for all the machines}{../../binpkg} !

	\item \htmladdnormallink{The demos: GL4Java2.4.0.0-demos-v1.zip}{../../archive/} The demos (sources and classes) for you to enjoy !

	\item \htmladdnormallink{The source: GL4Java2.4.0.0-src.tar.gz}{../../archive/} All sources no precompiled binaries no demos !

	\item \htmladdnormallink{The docs: GL4Java2.4.0.0-doc.zip}{../../archive/} This Documentation (html$+$postscript) plus javadoc-package-documentation !

\end{itemize}

\subsection{GL4Java's Version Numbers}
\label{versions}
Here is a little descriptions what the versions mean.

\begin{verbatim}
	binpkg/gl4javaX.Y.Z.R.zip	               (the JAR file)
	binpkg/libGL4JavaX.Y.Z.R-<UNIX-TYPE>.tar.gz    (the native libs - unix)
	binpkg/libGL4JavaX.Y.Z.R-<WIN32-TYPE>.zip      (the native libs - win32)

        archive/GL4JavaX.Y.Z.R-demos.zip  (demos      - for users :-)
        archive/GL4JavaX.Y.Z.R-src.tgz    (everything - for developers :-)
        archive/GL4JavaX.Y.Z.R-doc.tgz    (everything - for readers :-)

        (where X is the major, Y the minor, Z the bugfix and R the release 
	 library version number ! 
	 Take a look at VERSIONS.txt for a closer description !
	)

\end{verbatim}

\subsubsection{Version Compatibility}
\input{../VERSIONS.txt}

\section{GL4Java's Directory structure}
\label{dirstruct}

\bigskip

\latex{\begin{minipage}{15cm}}
\setcounter{footnote}{0}
\setcounter{mpfootnote}{0}

\begin{center}

\begin{tabular}{ll}
{\tt ./ } & this is the directory, \\
		        	&		where you extract this package\\
 & \\
{\tt ./GL4Java } & GL4Java root directory \\
 & \\
{\tt ./GL4Java/gl4java }        	&		java-package\footnote[1]{within the src-package}\\
{\tt ./GL4Java/gl4java/awt }        	&		java-package\footnotemark[1]\\
{\tt ./GL4Java/gl4java/jau/awt }       	&		java-package\footnotemark[1]\\
{\tt ./GL4Java/gl4java/system }       	&		java-package\footnotemark[1]\\
 & \\
{\tt ./GL4Java/sun/awt/macintosh }	&		additional java-package\footnotemark[1]\\
{\tt ./GL4Java/sun/awt/motif }		&		additional java-package\footnotemark[1]\\
{\tt ./GL4Java/sun/awt/windows } 	&		additional java-package\footnotemark[1]\\
 & \\
{\tt ./GL4Java/CNativeCode }   	&		native code\footnotemark[1]\\
 & \\
{\tt ./GL4Java/demos }			&		GL4Java demos\footnote[2]{within the demo-package}\\
{\tt ./GL4Java/demos/natives }		&		same demos as native implementation\footnotemark[2]\\
 & \\
{\tt ./GL4Java/mklibs }		&		shell-scripts to do dynamic libs\\
				&		for different plattforms\footnotemark[1] \\
				& \\
{\tt ./GL4Java/binpkg }		&		the precompiled binaries\footnote[3]{for the binary installation}\\
{\tt ./GL4Java/binpkg/gl4javaX.Y.Z.R.zip } 	&	GL4Java Java-Archive\footnotemark[3]\\
 & \\
{\tt ./GL4Java/binpkg/libGL4JavaX.Y.Z.R-<WIN32-TYPE>.zip } &  ... for Windows 32\footnotemark[3]\\
{\tt ./GL4Java/binpkg/libGL4JavaX.Y.Z.R-<UNIX-TYPE>.tar.gz } & ... for Unices\footnotemark[3]\\
{\tt ./GL4Java/CClassHeaders }		&	temporaery dir for javah\footnotemark[1] \\
 & \\
{\tt ./GL4Java/C2J }			&	the C2J cross-compiler\footnotemark[1] \\
				&	used to create OpenGL and GLU wrappers \\
{\tt ./GL4Java/C2J/manual }		&	manual wrappers (see above)\footnotemark[1] \\
 & \\
{\tt ./GL4Java/docs }			&	documentation target dir\footnote[4]{within the doc-package} \\
{\tt ./GL4Java/docs/html }		&	the html-documentation\footnotemark[4] \\
{\tt ./GL4Java/docs/html/icons }	&	icons for latex2html\footnotemark[4]\\
{\tt ./GL4Java/docs/html/images }	&	icons for javadoc\footnotemark[4]\\
 & \\
{\tt ./GL4Java/docs-src	 }	&	doc source for tex-doc\footnotemark[1] \\
{\tt ./GL4Java/archive }		&	all archives (*.tar.gz) resides here\footnote[5]{not within any package} ! \\
 & \\
{\tt ./GL4Java/Win32BC5 }		&	the target dir for Win32 development\\
				&	(Borland C Version 5.0)\footnotemark[1] \\
{\tt ./GL4Java/Win32VC5 }		&	the target dir for Win32 development \\
				&	(MS Visual C Version 5.0)\footnotemark[1] \\
\end{tabular}

\end{center}

\latex{\end{minipage}}


\section{Step by Step}

You can choose between an automatic installation with the GL4Java Installer
and a manual installation !

\subsection{\htmladdnormallink{GL4Java Installer}{../../INSTALLER.txt}}
\label{GL4JavaInstaller}
\begin{latexonly}
\input{../INSTALLER.txt}
\end{latexonly}

\subsection{\htmladdnormallink{Manual Binary Installation}{../../INSTALL.txt}}
\label{BinInstall}
\begin{latexonly}
\input{../INSTALL.txt}
\end{latexonly}

\subsubsection{\htmladdnormallink{Java2 Installation}{../../Java2.txt}}
\label{Java2Install}
\begin{latexonly}
\input{../Java2.txt}
\end{latexonly}

\subsubsection{\htmladdnormallink{Netscape Installation}{../../Netscape.txt}}
\label{NetscapeInstall}
\begin{latexonly}
\input{../Netscape.txt}
\end{latexonly}

\subsubsection{\htmladdnormallink{Microsoft JVM Installation}{../../MS-JVM.txt}}
\label{MSJVMInstall}
\begin{latexonly}
\input{../MS-JVM.txt}
\end{latexonly}

\subsection{Source Installation}
\label{srcinstall}

\subsubsection{Prerequisites}

You must allready have the following installed:


\latex{\begin{minipage}{15cm}}
\setcounter{footnote}{0}
\setcounter{mpfootnote}{0}

\begin{enumerate}


	\item Common

	\begin{enumerate}

	    \item JDK $\geq$  1.1.7 
	    \item JavaCC\cite{JavaCC} (chapter \ref{devenv}, page \pageref{devenv})
	    \item bash (chapter \ref{devenv}, page \pageref{devenv})
	    \item GNU make (chapter \ref{devenv}, page \pageref{devenv})
	    \item GNU tools (chapter \ref{devenv}, page \pageref{devenv})
	    \item Netscape Communicator 4.5 (optional)

	\end{enumerate}
	
	\item UNIX/X11

	\begin{enumerate}
	    \item ( GL + GLU ) or ( MesaGL + MesaGLU Version 2.4 or higher)
	      AND glut (only for c-demos yet)

	      We are looking for libGL.so AND libGLU.so in your library PATH,
	      so please create a symbolic link from the Mesa libs,
	      to the abstract one's !!

	    \item X11R6 (XFree86 works fine ;-)

	    \item Unix standard file-utilities (tar, gzip, ...)

	    \item LaTeX and LaTeX2Html\cite{Dragos96} (to generate the documentation)

	\end{enumerate}

	\item Windows 32 (NT \& 95)

	\begin{enumerate}
	    \item MS-OpenGL and GLU library installed - \emph{\bf must} !
	    	See if you have opengl32.dll AND glu32.dll 
		installed in your library path
		(c:/winnt/system32 OR c:/windows/system)

	    \item WinZip (to extract zip and tar-gzip files), be sure to disable the option LF/CR conversion in TAR-Archives ! You can also use the jar command-line tool of the JDK for zip files !

	    \item Unix standard utilities (make, ksh (bash), tar ...)
	    	E.g. GNU-Win32 Project from Cygnus\cite{GnuWin32}, make sure to have zip installed also !

	    \item LaTeX and LaTeX2Html\cite{Dragos96} (not testet - to generate the documentation)

	    \item For compiling GL4Java for the MS-JVM,
		  be sure you have the new MS-JVM machine (build 3186) installed.

	\end{enumerate}
    \end{enumerate}

\latex{\end{minipage}}

\subsubsection{Compilation}

	{\center{\bf UNICEs and WINDOWs users}}

You CAN (not a must) follow the Installation procedure above,
just be sure to copy the libs to your LD\_LIBRARY\_PATH
which should be the path in symbols.mak 'HOME\_LIB\_DIR' see below (UNICEs) !!
Also you should use a symbolic link OR edit CLASSPATH
to make 'GL4Java' and 'sun' visible in you CLASSPATH
(copy is not recommended - see below !!! ) !

You MUST follow the Installation procedure above,
if you want to USE GL4Java with JAVA2-JRE, JAVA2-Plug-In, 
Netscape 4.5 (Win32/Unix) JVM or MS-JVM/InternetExplorer.

Just procede the Installation procedure AFTER the compilation !

Login as root (or any other), 
obtain the source archive for GL4Java. 

copy the .tar.gz file to a directory such as /usr/local and 
cd to that directory, then unpack the file: 

\begin{verbatim}
   cd /usr/local	( or where ever you placed the file )
   gunzip GL4JavaX.Y.Z-Rn-src.tar.gz ; tar xvf GL4JavaX.Y.Z-Rn-src.tar
   gunzip GL4JavaX.Y.Z-Rn-doc.tar.gz ; 
   tar xvf GL4JavaX.Y.Z-Rn-doc.tar (optional :-)
\end{verbatim}

X is the major, Y the minor and Z the bugfix library version number,
and Rn (e.g. R2) is the release number, where no 
changes in the libs where made !

A new directory will be created called GL4Java.

Go to the directory GL4Java, 
which is the root-directory of GL4Java - so we stay here for the following 
procedures !

You will see some 'symbols.mak.$<$machine$>$' files.
Choose the right 'symbols.mak.$<$machine$>$' file, or create one.
Some older actual non-maintained files can be found under symbols.mak-old !
Edit your choosen or created 'symbols.mak.$<$machine$>$', so it will fit to
your OS configuration - all macros !
Copy the file 'symbols.mak.$<$your-machine$>$' to 'symbols.mak'.
Do not forget to add your 'HOME\_LIB\_DIR' (set in symbols.mak) to your
LD\_LIBRARY\_PATH (UNICEs). The created lib will be copied to 'HOME\_LIB\_DIR'
by automatic (via tar - symbolic links are still alive ;-) !

Next - you have to be sure, that the root-directory of GL4Java,
where the directories \emph{gl4java} and \emph{sun} exists,
is in your CLASSPATH.
E.g.: Assume you have expanded the package in the directory
\emph{/usr/local}.

\begin{verbatim}
	
	For sh/ksh:    CLASSPATH=$CLASSPATH:/usr/local/GL4Java
	   or
	for csh:	set CLASSPATH=(/usr/local/GL4Java $CLASSPATH)
	  or
	for WINDOW's  check out your GUI's

\end{verbatim}
This is evt. obsolete, if you have added the THISDIR path in the CLASSPATH 
of the symbols.mak's JAVAC definition, look into these files !
You must also use the macro 'DEST\_CLASSES\_DIR' (symbols.mak), where all
generated class-files will be copied into, when created.
You can also invoke �make classcpy� to force copy all class-files to 
'DEST\_CLASSES\_DIR'. Also the gl4java.jar file is generated there !

Because GL4Java supports Netscape's JVM with JNI, we do try
to ask for a privilege to run native DLL's.
Because this ask - the 'netscape' package is included in the file 
\emph{capsapi\_classes.zip} which is used in the \emph{JAVAC} macro definition
in \emph{symbols.mak}.

This is evt. obsolete, if you have entered the file 'capsapi\_classes.zip'
in the CLASSPATH of the symbols.mak's JAVAC definition, look into these files !

	{\center{\bf UNICEs users}}

Next create the library, class-files and all is needed with (for Unix/X11):

\begin{verbatim}
    make x11
\end{verbatim}

The Unix makefile-action does the complete creation :-)
The default shared-library files for Unice's are :
\begin{verbatim}
	  	GLContext:	libGL4JavaJauGljJNI.so
		GLFuncJauJNI:	libGL4JavaJauGLJNI.so
		GLUFuncJauJNI:	libGL4JavaJauGLUJNI.so

		This *.so files stands for the versions files
		with all symbolic links, e.g.:

		  libGL4JavaJauGljJNI.so     -> libGL4JavaJauGljJNI.so.2.1
		  libGL4JavaJauGljJNI.so.2.1 -> libGL4JavaJauGljJNI.so.2.1.2
		  libGL4JavaJauGljJNI.so.2.1.2
			
\end{verbatim}
		
They are moved to the HOME\_LIB\_DIR (see above)
or copied with the "-a" option to keep the symbolic links alive !
                            

	{\center{\bf WINDOWs users}}

Next create the class-files and all preparations with (Win32):

\begin{verbatim}
    make w32
\end{verbatim}

The Windows makefile-action only creates the tool (gljni) and
all Java classes and C-wrappers.
An additional C compiler invocation must be done (see \ref{WinCompile},
page \pageref{WinCompile}) to create the 
default shared-library files:
\begin{verbatim}
	  	GLContext     (SUN-JNI):	GL4JavaJauGljJNI.dll
		GLFuncJauJNI  (SUN-JNI):	GL4JavaJauGLJNI.dll
		GLUFuncJauJNI (SUN-JNI):	GL4JavaJauGLUJNI.dll
	  	GLContext     (MS-JVM) :	GL4JavaGljMSJDirect.dll
\end{verbatim}

To run GL4Java within MS-JVM and InternetExplorer, 
please see \ref{MSJVMInstall}, page \pageref{MSJVMInstall}.

	{\center{\bf UNICEs and WINDOWs users}}

To run GL4Java within Netscape, 
please \ref{NetscapeInstall}, page \pageref{NetscapeInstall}.

For a complete description of the makefile invokations see \ref{makeinvoke},
page \pageref{makeinvoke}.

Next create the demos with.

\begin{verbatim}
   cd demos
   make
\end{verbatim}

   If all goes well, just type e.g.: 'java glDemosCvs'
   in this directory, and
   you should see a demo manager for all the GL4Java demos we have written.

   To allow JAVA2's appletviewer or Java2-Plug-In using JNI native libs,
   just look at GL4Java/demo/Java2Applet.bat 
   (You have to change the gl4java.policy file) !

   You can - of course - use your 'policytool',
   which is shipped with JAVA2 !

   To use the demo Applets for Java2-Plug-In from www.jausoft.com,
   please add the following lines to your java.policy file:

\begin{verbatim}
	grant codeBase "http://www.jausoft.com/Files/Java/1.1.X/GL4Java/demos/-"
	{
	  permission java.security.AllPermission;
	};
\end{verbatim}

   You can also compare the totally native glut version 
   against the java version.
   The glut versions do exist in the demos path also (under native).

\subsection{Windows Source Installation\ddag}
\label{WinCompile}

To compile GL4Java, we still use MS Visual C++ 6.0 !
We also assume that Windows is installed under c:/WIN\_NT !
The compiler flags are set to Pentium Pro (686 / PII) 
with the optimizimg Intel compiler !

Also Java2 SDK is installed under c:/java2 !
Also MS Java SDK is installed under c:/MSJAVASDK !

Please check all location in the makefile !

\latex{\begin{minipage}{15cm}}
\setcounter{footnote}{0}
\setcounter{mpfootnote}{0}

\begin{enumerate}
	\item If you have installed cygwin32 and it's bash and make, etc...
	  AND set it up very well (CLASSPATH, PATH, ...)
	  you can invoke the makefile\footnote[2]{in the GL4Java directory, the main one} with {\tt make w32 }
	  to create all java-depended stuff and the C-wrappers.

	\item Befor going any further, check if the files
	  opengl32.dll glu.dll\footnotemark[1], glu32.dll exist
	  in c:/WIN\_NT/system32 OR c:/windows/system !
	  IF not get MS-OpenGL Lib (opengl32.dll, glu32.dll) !

	\item Go with the Explorer to the directory Win32VC6
	      and open the workspace Win32VC6.dsw !

	\item Now, you should fix some option in the project file,
	  we used:

	  The order of this path is important !

	  INCLUDEPATH:
	  \begin{verbatim}
		  c:/projects/java-1.1.X/GL4Java/CClassHeaders; \
		  c:/projects/java-1.1.X/GL4Java/CNativeCode; \
		  c:/java2/include; \
		  c:/java2/include/windows \
		  <the MSVC60 SDK PATH's ...>
	  \end{verbatim}

	  LIBPATH:
	  \begin{verbatim}
		  c:/WIN_NT/system32; \
		  <the MSVC60 SDK PATH's ...>
	  \end{verbatim}

	  WHERE:
	  Java2 is installed under {\tt c:/java2 }
	  GL4Java is installed under {\tt d:/projects/java-1.1.X/GL4Java }

	\item Now just compile each project and the targets are 
	      placed in the lib directory !

	\item Copy the generated dll's (in the libs-directory)
	      to your windows system32 directory,
	      or (better) to your /java2/jre/bin directory .

\end{enumerate}

\latex{\end{minipage}}

\section{Hacking}

For a complete description of the makefile invokations see \ref{makeinvoke},
page \pageref{makeinvoke}.

You can call 'make x11' for Unix/X11, or 'make w32' for Windows or 'make mac' for Macintosh to create all wrapper classes, Java classes, etc. For Unix/X11 users the native libraries are created also.

If you want to modify the sources, you have to edit the \*.skel files !!!
E.g.: You should modify 'gl4java/GLContext.java.skel' instead of 'gl4java/GLContext.java', because 'gl4java/GLContext.java' will be generated !!!!
Check the 'makefile' for more details !

{\bf Again:} Look for *.skel files and modify these, because they will be
taken to create the target source files without the prefix .skel !!!

Documentation for sources will be created later !!
Check \htmladdnormallink{gl4java.GLContext.html}{gl4java.GLContext.html}.

For a \emph{step by step} instruction, please see chapter \ref{srcinstall} on page \pageref{srcinstall}.

\chapter{Design\dag}

Since Version 2.0.0 R1 the object model of GL4Java is completly changed !
But it is very easy to port existing 1.X.X RX GL4Java stuff to the new 
object model. The script \htmladdnormallink{../../demos/change2GL4JavaV2.sh}{change2GL4JavaV2.sh} will help.

GL4Java is designed to use the native OpenGL library, which should be installed
on each platform. Instead of writing an own OpenGL library, we just write a wrapper to access the nativ platform dependend OpenGL library.

\section{Native-Call to OpenGL}
Native programs are written in a programing language, where the compiler
is able to produce binary code for the platforms processor.
These OpenGL programs will use the platforms OpenGL library.
\emph{Use} means, that they load the library and call the procedures which 
exists in the loaded library.

\section{Java-Call to OpenGL}
The java program should act like a native program. Java programs should use
the native OpenGL library on the specific platform.
The java program must load the wrapper library, which will load the OpenGL library. Like the OpenGL library, the wrapper library is platform depended and will be load in runtime, a so called dynamic library.
The platform independed java OpenGL application needs the platform depended wrapper library and all librarys the wrapper library needs, e.g.: the OpenGL library etc. .
To run a java OpenGL application, the user must have the OpenGL library installed and must install the wrapper library - that's all.
The java OpenGL applications have the performance like an native OpenGL application. The only reason for a performance less than the native one are the costs for the java procedures (to modify data etc.). The passing through of data from java to OpenGL and vice versa needs no special converting features, thus the passing does not consume much time.


\subsection{JNI to OpenGL}

\subsubsection{Data Type Mappings}
Because java should pass it�s data without any time consuming convertion,
we need a data type mapping style, where the OpenGL data types\cite{Woo97} fits in the 
java types and vice versa.
The first and important criteria for a type match is the size of bytes.
The secound and convinient criteria is to find aequivalent types in java,
where less or no data converting must be done by the wrapper library and the 
java OpenGL programmer himself.
Except for the marked OpenGL types we found a fine type matching which
fullfill our needs. All \emph{unsigned} types must match in a byte-size 
java aequivalent, so a special java-casting must be done by the java OpenGL programmer. Because java programmer uses arrays, we generate the GL4Java function for all Java-Array-Types, to achieve compatibility with the \emph{GLvoid *} 
(see below).

OpenGL Data Types Mapping to Java Types.
\label{datamap}

\bigskip

\latex{\begin{minipage}{15cm}}
\setcounter{footnote}{0}
\setcounter{mpfootnote}{0}

\begin{center}
\begin{tabular}{lllll}
\hline
Data Type & C-Type & JNI-Type & Java-Type & OpenGL Type \\
\hline
8-bit integer & unsigned char & jboolean & boolean & GLboolean \\
 & & & & \\
8-bit integer & signed char & jbyte & byte & GLbyte \\
 & & & & \\
8-bit unsigned integer & unsigned char & jbyte\footnote[1]{No Java matching for unsigned, using extra conversion} & byte\footnotemark[1] & GLubyte, \\
 & & & & \\
16-bit integer & short & jshort & short & GLshort \\
 & & & & \\
16-bit unsigned integer & unsigned short & jshort\footnotemark[1] & short\footnotemark[1] & GLushort \\
 & & & & \\
32-bit integer & int or long & jint & int & GLint, \\
 & & & & GLsizei \\
 & & & & \\
32-bit unsigned integer & unsigned int or & jint\footnotemark[1] & int\footnotemark[1] & GLuint, \\
                        & unsigned int & & & GLenum, \\
 & & & & GLbitfield \\
 & & & & \\
32-bit floating-point & float & jfloat & float & GLfloat, \\
 & & & & GLclampf \\
 & & & & \\
64-bit floating-point & double & jdouble & double & GLdouble, \\
 & & & & GLclampd \\
 & & & & \\
32-bit integer & void * & jbyteArray\footnote[2]{No Java matching for pointers, using any arrays} & byte[]\footnotemark[2] & GLvoid * \\
 & & jshortArray\footnotemark[2] & short[]\footnotemark[2] & GLvoid * \\
 & & jintArray\footnotemark[2] & int[]\footnotemark[2] & GLvoid * \\
 & & jfloatArray\footnotemark[2] & float[]\footnotemark[2] & GLvoid * \\
 & & jdoubleArray\footnotemark[2] & double[]\footnotemark[2] & GLvoid * \\
 & & jbooleanArray\footnotemark[2] & boolean[]\footnotemark[2] & GLvoid * \\
 & & jlongArray\footnotemark[2] & long[]\footnotemark[2] & GLvoid * \\
\end{tabular}
\end{center}

\latex{\end{minipage}}


\subsubsection{JNI Generation}
\label{jnidesign}

The wrapper library which is called by the java application is specified by 
the Java Native Interface (JNI)\cite{JavaTuto}. All native functions, which should be called by the java application must be declared as a native function in a java class.
E.g.:
\begin{verbatim}
public native void glClear ( int mask ) ;     
\end{verbatim}
The JNI compiler creates a header file for the C programming language, e.g. \emph{foo.h}.
This header file can be copied as an template to the C definition file, e.g. \emph{foo.c}. This definition file can modified to add the needed function bodies.
The function bodies, the functions, only has to call its C function out of the OpenGL library with a standard data type convertion.

In the begining we wrote some of the OpenGL wrapper functions by ourself,
but later we learned that we can not guarante that all functions are wrapped and that all functions were almost bug free.

To simplify the writing of the apropiate wrapper functions, we created a 
C-Header-File to Java-Native-Declarations plus C-Wrapper-Functions
generator. We named this generator \emph{C2J}.

Details see chapter \ref{c2jimpl} on page \pageref{c2jimpl}.

\section{OpenGL's Connection to the Windowing-System}
\label{windesign}
OpenGL is specified as a 3D rendering machine. To view the rendering results,
we need an visual output device, e.g. the monitor ;-).
To simplify the monitor output OpenGL uses an existing operating system, 
which allows graphical output on the screen. The X-Window-System or MS-Windows is a well known an often distributed graphical interface, where the X-Window-System is avaiable on many platforms.
The common tasks, means the minimum need, of a library which connects the OpenGL machine to the Windowing-System are:
\begin{itemize}
	\item connect the OpenGL machine to a window
	\begin{itemize}
		\item create a window
		\item connect the window handel to the OpenGL handle/context
		\item set up the window properties, e.g. colors
		\item close the window
	\end{itemize}
	\item re-rendering with OpenGL when the window must be refreshed
\end{itemize}

See the java class implementation in chapter \ref{sunpack} on page \pageref{sunpack} to find the platforms window handle.

\subsection{X-Window-System}
OpenGL's interface to the X-Window-System (X)\cite{Barkakati97} is the GLX library.
The GLX library\cite{Woo97} can connect the OpenGL machine\footnote{a instance of the OpenGL machine, the OpenGL context - in detail} to X.
GLX supports all criteria specified above.

\subsection{MS-Windows}
OpenGL's interface to MS-Windows (MSW) is the WGL library.
The WGL library\cite{Fosner97} can connect the OpenGL machine\footnote{a instance of the OpenGL machine, the OpenGL context - in detail} to MSW.
MSW supports all criteria specified above.

\section{OpenGL's Connection to the AWT}
\label{glconawt}
The Abstract Window Toolkit (AWT) is the windowing system of the java machine. 
The AWT is included in the standard java package. The AWT for Unix and MS-Windows uses native functions of the windowing system, to take advantages of the existing windowing system.
To achieve the platform window-handle, we use the (since GL4Java Version 1.1.0)
the following classes:
\begin{itemize}
	\item sun.awt.DrawingSurface
	\item sun.awt.DrawingSurfaceInfo
	\item sun.awt.X11DrawingSurface (for X11/motif)
	\item sun.awt.Win32DrawingSurface (for Windows 32-bit)
\end{itemize}

Of course, these classes are not explicit for user purposes,
but they do provide us with interfaces to achieve the needed native window handle. We got this technique from Java3D's Canvas3D (decompiled it ...).
You can see this implementation in:
\begin{itemize}
	\item sun/awt/motif/X11HandleAccess.java
	\item sun/awt/window/Win32HandleAccess.java
\end{itemize}

\section{OpenGL Mapping to Java}
The JNI and data-type mapping (see \ref{datamap} on page \pageref{datamap})
is allready specified above.

What still need to be specified is the place of the java OpenGL native 
function declarations and which OpenGL librarys should be wrapped.

\subsection{Convinient Placement}
Because OpenGL is not an object-orientated API, OpenGL programmers does not
use methods. To support the same OpenGL environment, we will use a java class,
which contains:
\begin{itemize}
	\item window control
	\item OpenGL functions
\end{itemize}

\subsection{Keeping the OpenGL C-API}

\subsubsection{The Naming Convention}
The GL4Java interface supports the OpenGLFrame class, 
which contains all OpenGL and GLU functions (glut in the near future :-) !
The naming convention of all gl* and glu* functions in the 
OpenGLFrame class is exactly the same as in the original,
so there are no problems in porting C code.
Datatypes java does not support will by supported by binary compatible types, 
e.g. for \emph{unsigned byte} in 
\emph{glColor4ubv}, GL4Java uses a \emph{byte}-array !


\chapter{Development Environment}
\label{devenv}
We use a directory structure as described above (see \ref{dirstruct}, page \pageref{dirstruct}).
Many well known Unix tools will be used, included some java tools.

\section{Unix Tools\dag}
We use the follwing standard tools as our develeopment environment:
\begin{enumerate}
	\item c-compiler and linker (GNU-C, ...)
	\item debugger (gdb, xxgdb)
	\item JavaCC\cite{JavaCC} (a java lex \& yacc parser generator)
	\item bash, make (GNU or other)
	\item tar \& gzip as our archive toolkit
\end{enumerate}

\section{Windows Tools\ddag}
All above named Unix tools are avaiable under MS-Windows 32 with the
GNU-Win32 project by Cygnus\cite{GnuWin32}.
We need an additional MS-Windows compiler, MC-Visual-C++ or Borland-C++
are appropiate MS-Windows compilers.
We took the Borland compiler, because of it's accessible,
but we guess it does not make a different.

\chapter{Implementation}
\label{implementation}

\section{Makefile\dag}
\label{makefile}
To manage the project, we wrote a makefile (used with make).
Our makefile is able to generate the complete code, the demos, the documentation and the archived packages.
The makefile resides in the projects root directory and is splitten
into two files:
\begin{itemize}
	\item makefile (where all common rules resides)
	\item symbols.mak (where all platform depended rules and macros resides)
\end{itemize}

A symbols.mak\* file must be created or copied \& edited for each platform
and linked to symbols.mak. We support some symbols.mak files like:

\begin{itemize}
	\item symbols.mak.aix
	\item symbols.mak.linux
	\item symbols.mak.solaris
	\item symbols.mak.win32
\end{itemize}

\subsection{Makefile Invokation}
\label{makeinvoke}

Here is the part of the makefile, which describes the different
actions, which are possible !

\begin{verbatim}
#
# for general creation (java + native-lib) invoke:
#
# make x11	: create java and native lib for unix/x11
# make w32	: create java and native lib for windows32
# make mac	: create java and native lib for Macintosh
#
#
# to copy the class-files to DEST_CLASSES_DIR, invoke:
#
# make classcpy
#
# 
# to save the native-library invoke after general creation:
#
# make unix2binpkg	: put the created unix-lib to the binpkg-dir
# make win2binpkg	: put the created win-lib to the binpkg-dir
# make mac2binpkg	: put the created macintosh-lib to the binpkg-dir
#
#
# to create the complete html documentation invoke
#
# make htmldoc		: unix
# make javadoc		: unix (javadoc only)
# make htmldocw32	: win32
# make javadocw32	: win32 (javadoc only)
#
#
# to put all together as an tar-gzip archive in the archive-dir, invoke:
#
# make archiv
#
#
# for cleanup (without archive and binpkg !!) invoke:
#
# make clean		: only temp-files (java, native)
# make cleannative	: only temp-files (native)
# make cleanall		: all temp-,java-,and native-files
#
\end{verbatim}

\subsection{Make-Scripts to generate the librarys\dag}

To support a platform independed creation for the wrapper libraries for
the different Unix platforms, we wrote a shell script therefor,
which will be invoked by the makefile.

The following files contains the apropiate scripts:

\latex{\begin{minipage}{15cm}}

\begin{center}
\begin{tabular}{ll}
\tt mklibs/mkexp.aix &  \\
\tt mklibs/mklib.aix &  \\
\tt mklibs/mklib.linux &  \\
\tt mklibs/mklib.solaris &  \\
\tt mklibs/mkslib.aix &  \\
\tt mklibs/mkslib.linux &  \\
\tt mklibs/mkslib.solaris &  \\
\end{tabular}
\end{center}

\latex{\end{minipage}}


\section{C2J Version 1.0\dag}
\label{c2jimpl}
As decribed in chapter \ref{jnidesign} on page \pageref{jnidesign}, we use C2J as our OpenGL wrapper functions generator.
C2J is our JNI and MSJDirect file generator.  C2J creates the java native function declarations and the C functions. C2J gets a modified Mesa-OpenGL header file as it's input
and generates the java native declarations in one file, and the complete C functions in another file. 
C2J is written in an Extended Backus Naur Form (EBNF) of the JavaCC\cite{JavaCC}parser generator.
We modified the C parser provided with the JavaCC to write C2J.

The C2J syntax can seen in \htmladdnormallink{GL4Java/C2J/C2J.jj}{./C2J.html} !

The C2J output files are glued with the other parts needed by the java class
and the C definition-file to the complete sources.

The new C2J Version supports the following new features:
\begin{enumerate}
\item Only the glGetString, gluGetString and gluErrorString functions must be written manualy now !!!!

\item Better compatibility with \emph{GLvoid *} arguments.
  This means 
\begin{verbatim}
	"byte[]", "short[]", "int[]",
	"float[]", "double[]", "boolean[]", "long[]"
\end{verbatim}
  functions are generated !

\item Better MS-JDirect (MS-JVM) integeration !
\end{enumerate}


The source files are:

\latex{\begin{minipage}{15cm}}
\setcounter{footnote}{0}
\setcounter{mpfootnote}{0}

\begin{center}
\begin{tabular}{ll}
\tt C2J/C2J.jj & the parser definition (input for JavaCC) \\
\tt C2J/makefile	 &			the lokal makefile \\
\tt C2J/gl-enum-auto.orig	 &		the OpenGL enumeration part, C2J input  \\
\tt C2J/gl-proto-auto.orig	 &		the OpenGL declaration part, C2J input  \\
\tt C2J/glu-enum-auto.orig &   the GLU enumeration part\\
\tt C2J/glu-proto-auto.orig & the GLU declarartion part \\
\tt C2J/manual/gl-manualCoded.orig & the GL manual coded C2J input \\
\tt C2J/manual/glu-manualCoded.orig &the GL manual coded C2J input part \\
\tt C2J/manual/gl-enum-manualCoded.java & the GL manual coded java part \\
\tt C2J/manual/glu-enum-manualCoded.java & the GLU manual coded java part \\
\tt C2J/manual/glu-manualCodedImplJNI.c & the GLU manual coded c part \\
\tt C2J/manual/glu-manualCodedImplJNI.h & the GLU manual coded c part \\
\tt C2J/manual/glu-manualCodedImplJNI.java & the GLU manual coded java part \\
\tt C2J/manual/glu-manualCodedImplMSJVM.java & the GLU manual coded java part \\
\tt C2J/manual/glu-manualCodedVirt.java  & the GLU manual coded java part \\
\tt C2J/manual/gl-manualCodedImplJNI.c & the GL manual coded c part \\
\tt C2J/manual/gl-manualCodedVirt.java & the GL manual coded java part \\
\tt C2J/manual/gl-manualCodedImplJNI.java & the GL manual coded java part \\
\tt C2J/manual/gl-manualCodedImplMSJVM.java & the GL manual coded java part \\
\end{tabular}
\end{center}

\latex{\end{minipage}}



\section{GLEnum}
\label{glenum}

Since Version 2.0.0, GLEnum is the interface, where all C-API OpenGL enumerations are defined as static final.
This class is generated, for details see chapter \ref{c2jimpl} on page \pageref{c2jimpl}.

\section{GLUEnum}
\label{glenum}

Since Version 2.0.0, GLUEnum is the interface, where all C-API GLU enumerations are defined as static final.
This class is generated, for details see chapter \ref{c2jimpl} on page \pageref{c2jimpl}.

\section{GLFunc}
\label{glfunc}

Since Version 2.0.0, GLFunc is the interface, where all C-API OpenGL functions are specified.
This class is almost generated, for details see chapter \ref{c2jimpl} on page \pageref{c2jimpl}.

\section{GLFuncJauJNI}

Since Version 2.0.0, GLFuncJauJNU is one implementation of the interface GLFunc.
This class is almost generated, for details see chapter \ref{c2jimpl} on page \pageref{c2jimpl}.

\section{GLUFunc}
\label{glufunc}

Since Version 2.0.0, GLUFunc is the interface, where all C-API GLU functions are specified.
This class is almost generated, for details see chapter \ref{c2jimpl} on page \pageref{c2jimpl}.

\section{GLUFuncJauJNI}

Since Version 2.0.0, GLUFuncJauJNU is one implementation of the interface GLUFunc.
This class is almost generated, for details see chapter \ref{c2jimpl} on page \pageref{c2jimpl}.

\section{GLContext}
\label{glcontext}

Since Version 2.0.0, GLContext is the central java class for the OpenGL bindings and it is defined as a package member of the java package gl4java.
GLContext trys to connect the OpenGL machine to the AWT window of
the given Component, which is represented by the platform specific window.

GLContext also can loads all the needed libraries and GLFunc and GLUFunc
implementations.

These are the basic source files, look at \ref{gl4javabasefiles}, page \pageref{gl4javabasefiles}.

\section{GLFrame}
\label{glframe}

Since Version 2.0.0, GLFrame is moved out !

\section{GLCanvas}
\label{glcanvas}

Since Version 2.0.0, GLCanvas is one AWT toolkit implementation for the OpenGL bindings and it is defined as a package member of the java package gl4java.awt.
GLCanvas is derived from AWT's Canvas. While creation GLCanvas 
trys to create a GLContext instance to connect the OpenGL machine to the AWT window, which is represented by the platform specific window.

Also the refresh signal is routed to OpenGL, so OpenGL can re-render.

These are the basic source files, look at \ref{gl4javabasefiles}, page \pageref{gl4javabasefiles}.

\section{GLAnimCanvas\dag\ddag}
\label{glanimcanvas}

Since Version 2.0.0, GLAnimCanvas is one AWT toolkit implementation for the OpenGL bindings and it is defined as a package member of the java package gl4java.awt.
GLAnimCanvas is derived from GLCanvas.
GLAnimCanvas addds thread support for animation, thats where the Anim stands for in GLAnimCanvas.

\latex{\begin{minipage}{15cm}}
\label{gl4javabasefiles}

\begin{center}
\begin{tabular}{ll}
\tt gl4java/GLEnum.java & The OpenGL Enumerates as an  interface. \\
\tt gl4java/GLUEnum.java & The GLU Enumerates as an  interface. \\
 & \\
\tt gl4java/GLFunc.java & The OpenGL functions interface. \\
\tt gl4java/GLUFunc.java & The GLU functions interface. \\
 & \\
\tt gl4java/GLFuncJauJNI.java & One OpenGL functions implementation. \\
\tt gl4java/GLUFuncJauJNI.java & One GLU functions implementation. \\
 & \\
\tt gl4java/GLContext.java & The OpenGL-Context Window service, and lib and class loader. \\
 & \\
\tt gl4java/GL4JavaInitException.java & Our Exception \\
 & \\
\tt gl4java/awt/GLCanvas.java & One OpenGL AWT binding  \\
\tt gl4java/awt/GLAnimCanvas.java & Another OpenGL AWT binding  \\
 & \\
\tt gl4java/jau/awt/WinHandleAccess.java & Interface to the platform's \\
                                         & windowing system  \\
\end{tabular}
\end{center}

\latex{\end{minipage}}


Java OpenGL programmers can derive their classes either from GLCanvas.
See the annotated Example in chapter \ref{olympicCvs} on page \ref{olympicCvs}.

See the Html-Source documentation for details (see \ref{GL4JavaHtml} on page \pageref{GL4JavaHtml}).

\section{Extensions for the sun-package\dag\ddag}
\label{sunpack}

Like described in chapter \ref{windesign} on page \pageref{windesign},
we need to find out the platforms window handle to connect it with the OpenGL machine.

The used motif and windows sub-packages are not defined as a standard,
but they exist as long Java 1.0.2 exist and they are the only way to
find out the unique AWT window's platform-window-handle from the java-side.

See the Html-Source documentation for details (see \ref{GL4JavaHtml} on page \pageref{GL4JavaHtml}).

Here are the apropiate source files:

\latex{\begin{minipage}{15cm}}

\begin{center}
\begin{tabular}{ll}
\tt sun/awt/motif/X11HandleAccess.java & X-Window-System glue \\
\tt sun/awt/windows/Win32HandleAccess.java & MS-Windows glue \\
\tt sun/awt/macintosh/MacHandleAccess.java & Macintosh glue \\
\end{tabular}
\end{center}
See also \ref{glconawt} on page \pageref{glconawt}.

\latex{\end{minipage}}

\section{The Native Wrapper-Library Code\dag\ddag}
\latex{\begin{minipage}{15cm}}

As described in chapter \ref{windesign} on page \pageref{windesign}
we need a connection between the platform depended window system and OpenGL.

Here are the apropiate source files:

\begin{center}
\begin{tabular}{ll}
\tt CNativeCode/OpenGL\_Win32.c & MS-Windows window access \\
\tt CNativeCode/OpenGL\_X11.c &  X-Window-System window access\\
\tt CNativeCode/OpenGL\_misc.c &  common miscelaneos sources \\
\tt CNativeCode/OpenGLU\_funcs.c.skel & common GLU source-part \\
\tt CNativeCode/OpenGLU\_funcs.c & glued source from the *.skel and C2J output \\
\tt CNativeCode/OpenGL\_funcs.c.skel & common GL source-part\\
\tt CNativeCode/OpenGL\_funcs.c & glued source from the *.skel and C2J output \\
\tt CNativeCode/gl.aix.not.c & Dummy GL functions not defined in AIX OpenGL library \\
\tt CNativeCode/gl.aix.not.h & Dummy GL functions not defined in AIX OpenGL library \\
\tt CNativeCode/gl.win32.not.c & Dummy GL functions not defined in MS-Windows OpenGL library \\
\tt CNativeCode/gl.win32.not.h & Dummy GL functions not defined in MS-Windows OpenGL library \\
\tt CNativeCode/winstuff.h & Miscelaneous declarations missing in MS-Windows \\
\end{tabular}
\end{center}

\latex{\end{minipage}}

See the Source documentation for details.


\chapter{Examples\dag\ddag}

\section{The olympicCvs GLAnimCanvas derivation}
\label{olympicCvs}
\begin{verbatim}
/**
 * @(#) olympicCvs.java
 */

import gl4java.awt.GLAnimCanvas;
import java.awt.*;
import java.awt.event.*;
import java.lang.Math;


class olympicCvs extends GLAnimCanvas
{
/**
 * Instead of using suspend (JAVA2)
 *
 * @see run
 */      
	protected boolean threadSuspended = true;

	static final double M_PI= 3.141592654;
	static final double M_2PI= 2*3.141592654;

	static final int 
		XSIZE=   100,
		YSIZE=   75,
		RINGS= 5,
		BLUERING= 0,
		BLACKRING= 1,
		REDRING= 2,
		YELLOWRING =3,
		GREENRING =4,
		BACKGROUND =8,
		BLACK = 0,
		RED=1,
		GREEN=2,
		YELLOW=3,
		BLUE=4,
		MAGENTA=5,
		CYAN=6,
		WHITE=7;

	byte rgb_colors[][];
	int mapped_colors[];
	float dests[][];
	float offsets[][];
	float angs[];
	float rotAxis[][];
	int iters[];
	int theTorus;

        float lmodel_ambient[] = {0.0f, 0.0f, 0.0f, 0.0f};
        float lmodel_twoside[] = {0.0f, 0.0f, 0.0f, 0.0f};
        float lmodel_local[] = {0.0f, 0.0f, 0.0f, 0.0f};
        float light0_ambient[] = {0.1f, 0.1f, 0.1f, 1.0f};
        float light0_diffuse[] = {1.0f, 1.0f, 1.0f, 0.0f};
        float light0_position[] = {0.8660254f, 0.5f, 1f, 0f};
        float light0_specular[] = {1.0f, 1.0f, 1.0f, 0.0f};
        float bevel_mat_ambient[] = {0.0f, 0.0f, 0.0f, 1.0f};
        float bevel_mat_shininess[] = {40.0f, 0f, 0f, 0f};
        float bevel_mat_specular[] = {1.0f, 1.0f, 1.0f, 0.0f};
        float bevel_mat_diffuse[] = {1.0f, 0.0f, 0.0f, 0.0f};

	public olympicCvs (int w, int h, 
			   String glClass, String gluClass )
	{
		super(w, h, glClass, gluClass );
	}

        public void init() 
	{
		rgb_colors=new byte[RINGS][3];
		mapped_colors=new int [RINGS];
		dests=new float [RINGS][3];
		offsets=new float[RINGS][3];
		angs=new float[RINGS];
		rotAxis=new float[RINGS][3];
		iters=new int[RINGS];

		int i;
		float top_y = 1.0f;
		float bottom_y = 0.0f;
		float top_z = 0.15f;
		float bottom_z = 0.69f;
		float spacing = 2.5f;

		ReInit();
		for (i = 0; i < RINGS; i++) {
		  rgb_colors[i][0] = rgb_colors[i][1] = rgb_colors[i][2] = (byte)0;
		}
		rgb_colors[BLUERING][2] = (byte)255;
		rgb_colors[REDRING][0] = (byte)255;
		rgb_colors[GREENRING][1] = (byte)255;
		rgb_colors[YELLOWRING][0] = (byte)255;
		rgb_colors[YELLOWRING][1] = (byte)255;
		mapped_colors[BLUERING] = BLUE;
		mapped_colors[REDRING] = RED;
		mapped_colors[GREENRING] = GREEN;
		mapped_colors[YELLOWRING] = YELLOW;
		mapped_colors[BLACKRING] = BLACK;

		dests[BLUERING][0] = -spacing;
		dests[BLUERING][1] = top_y;
		dests[BLUERING][2] = top_z;

		dests[BLACKRING][0] = 0.0f;
		dests[BLACKRING][1] = top_y;
		dests[BLACKRING][2] = top_z;

		dests[REDRING][0] = spacing;
		dests[REDRING][1] = top_y;
		dests[REDRING][2] = top_z;

		dests[YELLOWRING][0] = -spacing / 2.0f;
		dests[YELLOWRING][1] = bottom_y;
		dests[YELLOWRING][2] = bottom_z;

		dests[GREENRING][0] = spacing / 2.0f;
		dests[GREENRING][1] = bottom_y;
		dests[GREENRING][2] = bottom_z;

		theTorus = gl.glGenLists(1);
		gl.glNewList(theTorus, GL_COMPILE);
		FillTorus(0.1f, 8, 1.0f, 25);
		gl.glEndList();

		gl.glEnable(GL_CULL_FACE);
		gl.glCullFace(GL_BACK);
		gl.glEnable(GL_DEPTH_TEST);
		gl.glClearDepth(1.0);

		gl.glLightfv(GL_LIGHT0, GL_AMBIENT, light0_ambient);
		gl.glLightfv(GL_LIGHT0, GL_DIFFUSE, light0_diffuse);
		gl.glLightfv(GL_LIGHT0, GL_SPECULAR, light0_specular);
		gl.glLightfv(GL_LIGHT0, GL_POSITION, light0_position);
		gl.glEnable(GL_LIGHT0);

		gl.glLightModelfv(GL_LIGHT_MODEL_LOCAL_VIEWER, lmodel_local);
		gl.glLightModelfv(GL_LIGHT_MODEL_TWO_SIDE, lmodel_twoside);
		gl.glLightModelfv(GL_LIGHT_MODEL_AMBIENT, lmodel_ambient);
		gl.glEnable(GL_LIGHTING);

		gl.glClearColor(0.5f, 0.5f, 0.5f, 0.0f);

		gl.glMaterialfv(GL_FRONT, GL_AMBIENT, bevel_mat_ambient);
		gl.glMaterialfv(GL_FRONT, GL_SHININESS, bevel_mat_shininess);
		gl.glMaterialfv(GL_FRONT, GL_SPECULAR, bevel_mat_specular);
		gl.glMaterialfv(GL_FRONT, GL_DIFFUSE, bevel_mat_diffuse);

		gl.glColorMaterial(GL_FRONT_AND_BACK, GL_DIFFUSE);
		gl.glEnable(GL_COLOR_MATERIAL);
		gl.glShadeModel(GL_SMOOTH);

		gl.glMatrixMode(GL_PROJECTION);
		glu.gluPerspective(45f, 1.33f, 0.1f, 100.0f);
		gl.glMatrixMode(GL_MODELVIEW);
                glj.gljCheckGL();
	}

	public void display()
	{
	  int i;

	  /* Standard GL4Java Init */
	  if( glj.gljMakeCurrent(true) == false ) 
	  {
		System.out.println("problem in use() method");
		System.exit(0);
		return;
	  }

	  gl.glPushMatrix();

	  gl.glClear(GL_COLOR_BUFFER_BIT | GL_DEPTH_BUFFER_BIT);
	  glu.gluLookAt(0, 0, 10, 0, 0, 0, 0, 1, 0);

	  for (i = 0; i < RINGS; i++) {
	    gl.glColor3ubv(rgb_colors[i]);
	    gl.glPushMatrix();
	    gl.glTranslatef(dests[i][0] + offsets[i][0], dests[i][1] + offsets[i][1],
	      dests[i][2] + offsets[i][2]);
	    gl.glRotatef(angs[i], rotAxis[i][0], rotAxis[i][1], rotAxis[i][2]);
	    gl.glCallList(theTorus);
	    gl.glPopMatrix();
	  }

	  gl.glPopMatrix();
	  glj.gljSwap();
	  glj.gljCheckGL();
	  glj.gljFree();
	}

	public void animationCalc() 
	{
	          int i, j;

		  for (i = 0; i < RINGS; i++) {
		    if (iters[i]!=0) {
		      for (j = 0; j < 3; j++) {
			offsets[i][j] = Clamp(iters[i], offsets[i][j]);
		      }
		      angs[i] = Clamp(iters[i], angs[i]);
		      iters[i]--;
		    }
		  }
	          if (iters[0]==0)
		  {
			  shallWeRender=false;
		  }
	}

	public void ReInit()
	{
	  int i;
	  float deviation;

	  deviation = MyRand() / 2;
	  deviation = deviation * deviation;
	  for (i = 0; i < RINGS; i++) {
	    offsets[i][0] = MyRand();
	    offsets[i][1] = MyRand();
	    offsets[i][2] = MyRand();
	    angs[i] = (float) (260.0 * MyRand());
	    rotAxis[i][0] = MyRand();
	    rotAxis[i][1] = MyRand();
	    rotAxis[i][2] = MyRand();
	    iters[i] = ( int ) (deviation * MyRand() + 60.0);
	  }
	}

	public void FillTorus(float rc, int numc, float rt, int numt)
	{
	  int i, j, k;
	  double s, t;
	  float x, y, z;

	  for (i = 0; i < numc; i++) {
	    gl.glBegin(GL_QUAD_STRIP);
	    for (j = 0; j <= numt; j++) {
	      for (k = 1; k >= 0; k--) {
		s = (i + k) % numc + 0.5;
		t = j % numt;

		x = (float) Math.cos(t * M_2PI / numt) * (float) Math.cos(s * M_2PI / numc);
		y = (float) Math.sin(t * M_2PI / numt) * (float) Math.cos(s * M_2PI / numc);
		z = (float) Math.sin(s * M_2PI / numc);
		gl.glNormal3f(x, y, z);

		x = (rt + rc * (float) Math.cos(s * M_2PI / numc)) * (float) Math.cos(t * M_2PI / numt);
		y = (rt + rc * (float) Math.cos(s * M_2PI / numc)) * (float) Math.sin(t * M_2PI / numt);
		z = rc * (float) Math.sin(s * M_2PI / numc);
		gl.glVertex3f(x, y, z);
	      }
	    }
	    gl.glEnd();
	  }
	}

	public float Clamp(int iters_left, float t)
	{

	  if (iters_left < 3) {
	    return 0.0f;
	  }
	  return (iters_left - 2) * t / iters_left;
	}

	public float MyRand()
	{
	  // return 10.0 * (drand48() - 0.5);
	  return (float) ( 10.0 * (Math.random() - 0.5) );
	}

}
\end{verbatim}

\section{Further Examples\dag\ddag}
Please look in the directory demos to look at the samples.
You can start the glDemosCvs (gl4java.awt.GLAnimCanvas) java application to see them all running.

\chapter{Further Documentation\dag}

\subsection{\htmladdnormallink{Readme}{../../README.txt}}
\label{readme}
\begin{latexonly}
\input{../README.txt}
\end{latexonly}

\subsection{\htmladdnormallink{Changes}{../../CHANGES.txt}}
\label{changes}
\begin{latexonly}
\input{../CHANGES.txt}
\end{latexonly}

\subsection{\htmladdnormallink{License}{../../LICENSE.txt}}
\label{license}
\begin{latexonly}
\input{../LICENSE.txt}
\end{latexonly}

\section{Package-Documentation (javadoc)}
\label{GL4JavaHtml}
To look at the javadoc generated source documentation, go to
\htmladdnormallink{packages.html}{packages.html}.

Only the paperware produced for the FH-Bielefeld contains a printout of this pages as an extra appendix.

\section{Important Links}

At our homepage we generated some important links at
\htmladdnormallink{../../LINKS.html}{../../LINKS.html}.


\chapter{Resume}
\section{Contact us}
Like we described in the Foreword, we would be very thankfull for any response.
\begin{itemize}
 \item \htmladdnormallink{gl4java@jausoft.com}{mailto:gl4java@jausoft.com} 1st email adress
 \item \htmladdnormallink{a\_kolac@bielefeld.netsurf.de}{mailto:a_kolac@bielefeld.netsurf.de} 2nd email adress
 \item \htmladdnormallink{Homepage: http://www.jausoft.com}{http://www.jausoft.com}
\end{itemize}

\subsection{\htmladdnormallink{Thanxs}{../../THANXS.txt}}
\label{thanxs}
\begin{latexonly}
\input{../THANXS.txt}
\end{latexonly}

\appendix

\latex{%\documentclass[a4paper]{report}
%\usepackage{german}

%
%\begin{document}
%
%\begin{titlepage}
%  {\LARGE
%      \begin{tabular}{p{20mm}l}
%	      \multicolumn{2}{r}{\bf FH--Bielefeld}\\[4mm]
%	      \hline \\[3mm]
%	      & Einf"uhrung\\
%	      & f"ur\\
%	      & OpenGL \& Java\\[5mm]
%	      & Atilla Kolac\\[5mm]
%	      & Labor f"ur Parallelverarbeitung \\
%	      & FH-Bielefeld\\[5mm]
%	      \hline \\[20mm]
%	      & \today
%      \end{tabular}
%      }
%\end{titlepage}
%
%\tableofcontents
%

\chapter{OpenGL Einf"uhrung\ddag}

\section{Was ist OpenGL?}
OpenGL \cite{Woo97} \cite{Fosner97} wurde von Silicon Graphics,Inc.(SGI) entwickelt und ist eine verbreitete
Hardware- und Betriebsunabh"angige Bibliothek f"ur die Erstellung von 
dreidimensionalen Grafiken.
GL steht hierbei f"ur Grafics Library.

Au"serdem stellt OpenGL eine unabh"angige Programmierschnittstelle dar.
Wobei f"ur die Hardwareseite eine geeignete Implementation programmiert werden
kann. Die Softwareschnittstelle besteht aus ca. 120 Kommandos, die zur
Spezifikation von Objekten und Operationen ben"otigt werden.
OpenGL ist client- und serverf"ahig. Das Protokoll welches OpenGL-Befehle 
"ubermittelt, ist identisch. Dadurch k"onnen OpenGL-Programme
im Netzwerk ablaufen, in denen client und server Rechner verschiedenen
Typs sind. Dies erreicht man indem die Kommandos zum Ablauf von Windows-Tasks
oder zur Interaktion mit dem Benutzer nicht in dem OpenGL-Programm implementiert
werden. Sondern diese Aufgaben werden direkt mit der 
Fensteroberfl"achenprogrammierung der
entsprechenden Hardware mit"ubernommen.

Der Sinn der OpenGL-Bibliothek ist es, zwei- und dreidimensionale Objekte in 
einem Frame Buffer zu \emph{rendern}. Dieser ist mit einem Pixelspeicher in der Grafik-
Hardware eines PC vergleichbar. 
OpenGL ist streng prozedural. Das hei"st,
da"s nicht das Aussehen eines Objektes beschrieben wird sondern wie das 
Objekt gezeichnet wird. Hierzu bedient sich OpenGL zwei - oder dreidimensionalen
\emph{Vertices}. Diese repr"asentieren einen Punkt, z.B. den Endpunkt einer Linie 
oder eines Polygons. Die n"achste Ebene sind \emph{Primitive}, die aus einem oder
mehreren \emph{Vertices} bestehen.

Wie Vertices zu Primitiven zusammengesetzt werden und in ein Frame Buffer 
gezeichnet werden, wird durch eine Vielzahl von Einstellm"oglichkeiten 
kontrolliert.

Mit OpenGL kann man weiterhin Oberfl"achen zeichnen, Beleuchtungsspezifikationen
anwenden und Texturen verwenden.

Die wichtigsten Schritte bis zum Rendern einer Szene

\begin{enumerate}

\item Aus den geometrischen Primitiven werden Formen konstruiert und damit
mathematische Beschreibungen der Objekte erstellt. 
OpenGL sieht Punkte, Linien, Polygone, Bilder und Bitmaps als Primitive an.

\item Die Objekte werden im dreidimensionalen Raum arrangiert und der gew"unschte 
\emph{Blickpunkt} "uber der Szene wird ausgesucht.

\item Die Farben aller Objekte werden berechnet.
Zwar sind die Farben f"ur die Objekte im Programm definiert,
m"ussen jedoch neu berechnet werden falls sich das Objekt bewegt oder die 
Schattierung sich "andert.

\item Die mathematische Beschreibung von Objekten und den ihnen beigef"ugten
Farbinformationen werden in Bildschiermpixel konvertiert.
Dieser Proze"s wird \emph{Rasterung} genannt.

\end{enumerate}


Alle Daten, ob sie nun Geometrien oder Pixel beschreiben, k"onnen in einer
\emph{Display List} gesichert werden.
OpenGL unterscheidet zwischen zwei Berechnungsformen: 
Die Berechnung der Grafik wird sofort ausgef"uhrt (\emph{immediate-mode}).
Das bedeutet wenn das Kommando zum Zeichnen eines Objektes abgesetzt wird,
wird das Objekt gezeichnet. Dieser Modus ist in OpenGL voreingestellt.
Weiterhin lassen sich Kommandos in einer \emph{Display List} sichern, um sie sp"ater
auszuf"uhren.
Die sofortige Ausf"uhrung der Grafik l"asst sich einfacher programmieren,
das Sichern in einer Liste ist effizienter.

\section{Grundlagen}

\subsection{Client- Serverprotokoll}

\subsection{Datentypen}

\subsection{Beleuchtung}
OpenGL berechnet die Farben jedes Pixels, bevor die Szene dargestellt wird
und speichert sie im Frame Buffer ab. Dabei wird das in der Szene benutzte Licht
berechnet, und die Art, wie die Objekte in der Szene das Licht reflektieren 
oder absorbieren sollen.
Ein unbeleuchtetes dreidimensionales Objekt kann nicht von einem 
zweidimensionalen Objekt unterschieden werden.
Denn auf einer zweidimensionalen Fl"ache, sprich unser Bildschirm, 
wird die dreidimensionalit"at durch die Art der Beleuchtung und der
Schattierung erreicht.
Das zeigt, wie wesentlich die Beziehung zwischen Objekten und dem Licht 
in einer dreidimensionalen Szene ist.
OpenGL stellt Licht und Schatten durch Zerlegung der Objektoberfl"ach in
bestimmte rote, gr"une und blaue Komponenten dar.
Der Ort der Lichtquelle, sein Einstrahlungswinkel und Art m"ussen programmiert
werden.

Das Beleuchtungsmodell in OpenGL besteht aus vier Komponenten:
\emph{emitted, ambient, diffuse und specular}.
Alle vier Komponenten bewirken eine unterschiedliche Beleuchtungsart.
Sie werden getrennt angegeben und danach zusammenaddiert,
so das die Szene in das gew"unschte Licht gesetzt wird.

\section{Die Programmierung}

\subsection{Namenskonvention}

\subsection{Die Kommandosyntax}
OpenGL-Kommandos beginnen in der Programmiersprache C 
mit dem Pr"afix \emph{gl}, 
jedes Wort beginnt mit einem Gro"sbuchstaben ( z.B. glClearColor() ).
Konstanten beginnen mit dem gleichen Pr"afix, 
allerdings werden sie in Gro"sbuchstaben geschrieben und durch Unterstriche 
verbunden ( z.B. GL\_CLEAR\_BUFFER\_BIT ).
in dem Kommando \emph{glColor3f()} steht 3 f"ur die Anzahl der gegebenen Argumente,
f steht f"ur die Gleitkommazahlen.

Die ANSI-C-Implementation von OpenGL erm"oglicht bis zu 8 verschiedene
Datentypen.
Die Datentypen umfassen 8, 16 und 32-Bit-Integer, im signed- und unsigned-
Format und 32- oder 64-Bit-Gleitkommazahlen.
Einige OpenGL-Kommandos sind mit dem Endbuchstaben v versehen, was darauf hindeutet, da"s
das Kommando einen Zeiger auf einen Vektor oder Feld beinhaltet.

\subsection{Die Bibliotheken}
OpenGL bietet eine Anzahl leistungsf"ahiger und einfacher Kommandos zum Rendern.
Alle komplexen Zeichnungen m"ussen mit diesen Kommandos ausgef"uhrt werden.
Deshalb ist es m"oglich, eine eigene Bibliothek zu schreiben, die auf 
OpenGL aufsetzt. Damit l"a"st sich z.B. die Behandlung der Fenster vereinfachen.

Hier sind zwei sehr verbreite Bibliotheken aufgelistet:
\begin{itemize}
\item Die \emph{OpenGL Utility Library (GLU)} enth"alt einige Routinen die einfachere
OpenGL Kommandos benutzt. Es werden Funktionen zur Vereinbarung von Matrizen f"ur
besondere Betrachtungsorientierungen und Projektionen und zum Rendern von Oberfl"achen
definiert.
GLU-Funktionen sind durch den Pr"afix \emph{glu} zu erkennen.

\item Die OpenGL-Erweiterung f"ur das X-Windows System (GLX)  sieht die Erzeugung 
eines OpenGL-Kontextes vor und die Hinzuf"ugung eines Fensters, in dem gezeichnet
werden kann (drawable window), falls auf einem Rechner mit X-Window System 
gearbeitet wird.
GLX-Funktionen besitzen das Pr"afix \emph{glx}.
\end{itemize}
 
Hinzukommen noch eine gro"se Anzahl weiterer Bibliotheken, die wir hier 
verst"andlicherweise nicht aufz"ahlen k"onnen.


 
\subsection{Die OpenGL Windows-Programmierung}
Bevor die OpenGL-Bibliothek unter MS-Windows verwendet werden kann, m"ussen 
eine Reihe von Initialisierungsschritte ausgef"uhrt werden.
Jede Windows-OpenGL-Applikation mu"s seinen \emph{Rendering Context} ( OpenGL-Seite )
mit seinem \emph{Device Context} ( Windows-Seite ) verbinden. Hierzu mu"s zuerst die 
Win32-Funktion \emph{SetPixelFormat} und danach die Funktion \emph{wglCreateContext} mit
dem Device Context Handle Parameter aufgerufen werden. Ist dies erfolgreich,
gibt wglCreateContext einen Rendering Context Handle des Typs \emph{HGLRC} zur"uck.
OpenGL unter Windows kennt zwei Typen von Pixel-Modi:
Einen Modus, der die direkte Angabe von Pixelfarben erlaubt.
Und ein Modus, der die Auswahl der Farbe aus einer Palette unterst"utzt.

Es gibt spezielle Anforderungen durch OpenGL an ein Ausgabefenster.
Es mu"s mit den Fensterstilen \emph{WS\_CLIPSIBLINGS} und \emph{WS\_CLIPCHILDREN} initialisiert
sein um OpenGL Kompabilit"at zu gew"ahrleisten. 
Um die Perfomance der Applikation zu erh"ohen, sollte das Fenster einen
NULL-Hintergrund haben, da dieser sowieso durch die OpenGL-Bibliothek
gel"oscht wird.
Bevor ein Rendering Context verwendet werden kann, mu"s er mit der wglMakeCurrent
Funktion als der aktuelle Context bestimmt werden. Ist der Rendering Context
bereit um Kommandos entgegenzunehmen k"onnen weitere Initialisierung-Routinen
aufgerufen werden. Z.B. um den Frame Buffer zu l"oschen, Koordinatentransformationen
aufzusetzen, Lichtquellen zu konfigurieren oder andere Optionen zu setzen.

Ein solcher Initialisierungsschritt, der nicht ausgelassen werden darf,
ist der Aufruf der \emph{glViewport}-Funktion. Sie initialisiert oder modifiziert
die gr"o"se des\emph{ Rendering Viewports}. Typischerweise wird diese Funktion
ganz zu Anfang der Applikation aufgerufen und dann jedesmal wenn sie eine
\emph{WM\_SIZE}-Meldung erh"alt. Diese zeigt eine Gr"o"sen"anderung des Fensters an.


\chapter{Java Einf"uhrung\ddag}

Weitere englischsprachige Dokumentationen sind von Sun Microsystems erh"altlich\cite{JavaTuto}.

\section{Was ist Java?}
Sun Microsystems stellte Mitte 1995 sein Java Language Enviroment (kurz Java)
offiziell vor.

Java ist eine objektorientierte Programmiersprache und 
ist Plattformunabh"angig.

Java-Programme k"onnen entweder lokal auf einem Rechner ausgef"uhrt werden,
oder aber auch "uber das Internet gestartet werden.

\section{Grundlagen}

\subsection{Die Entwicklungsumgebung}
Sun Microsystems stellt f"ur die Plattformen Solaris(Sparc), Solaris(Intel) und Win32(Intel)\footnote{Windows 95 und Windows NT} die Java-Entwicklungsumgebung \emph{Java Development Kit (JDK)} kostenlos zur Verf"ugung. 
IBM stellt ihrerseits f"ur Windows 3.1(Intel), AIX(RS6000) und OS/2(Intel) die JDK zur Verf"ugung.

Die JDK enth"alt den Java-Compiler, die JVM und die standard Bibliothek, sowie
den Java-Debugger. Der Compiler als auch der Debugger sind Kommandozeilentools und enthalten keine GUI.
 
Mittlerweilen ist auch das \emph{Java Runtime Environment (JRE)} ver"offentlicht, welches lediglich die JVM und die standard Bibliothek enth"alt. Eigene Java-Applikationen k"onnen mit dem JRE geb"undelt und herausgegeben werden.

Der Java-Compiler generiert aus dem Java-Sourcecode den Java-Bytecode, wobei
dieser Java-Bytecode von der Java-Virtuellen-Maschiene (JVM) interpretiert wird.

Der Java-Sourcecode enth"alt die Klassendefinitionen.
In der Regel gibt es je Klassendefinition eine Quelldatei.
Der Dateiname der Quelldatei setzt sich zusammen aus dem Klassennamen und der Dateinamensendung \emph{.java}. Z.B. ist die Klassendefinition fuer \emph{Foo} in der Datei \emph{Foo.java} enthalten. Gross- und Kleinschreibung ist hierbei zu beachten.

Der Java-Compiler "ubersetzt die Quelldatei mit der Namensendung \emph{.java} in die Zieldatei f"ur den Java-Bytecode mit der Namensendung \emph{.class}.
Z.B. erzeugt der Aufruf von 

\begin{verbatim}
javac Foo.java
\end{verbatim}

die Datei \emph{Foo.class}.

Das compilierte Java-Programm, der Java-Bytecode, kann z.B. mittels

\begin{verbatim}
java Foo
\end{verbatim}

aufgerufen werden. Dieser Aufruf startet die JVM und beginnt das Programm
\emph{Foo} abzuarbeiten.

\subsection{Java-Virtuelle-Maschine (JVM)}
Die JVM simuliert eine komplette Betriebssystemumgebung
und ist in der Lage den Java-Bytecode abzuarbeiten.
Hiermit sind Java-Programme Plattform unabh"angig, da die JVM
auf verschiedenen Plattformen ein Einheitliches Java-System zur 
Verf"ugung stellen.

Mittels der JVM und den standard Java-Bibliotheken kann ein Java-Programm
\begin{itemize}
	\item auf das Netzwerk/Internet zugreifen
	\item eine grafische Oberfl"ache besitzen (AWT)
	\item auf native Funktionen zugreifen (JNI)
	\item auf Datenbanken zugreifen (jdbc)
	\item wiederverwendbare Komponente besitzen (Beans)
	\item gemeinsame standardisierte Objekte mit Java und 
	nicht-Java Applikationen austauschen (Corba/RMI)
	\item etc.
\end{itemize}

\subsection{Java-Interpreter und JIT-Compiler}
Der Java-Interpreter ist bestandteil der JVM. Der Java-Bytecode wird von dem Java-Interpreter abgearbeitet. Der JIT-Compiler (Just-In-Time-Compiler) ist eine Erweiterung des Java-Interpreters und "ubersetzt den Java-Bytecode blockweise in plattformspezifischen Machienencode.

\section{Java-Applets und Java-Applikationen}
Java-Programme lassen sich grob in zwei Kategorien teilen:
Java-Applikationen bezeichnen eigenst"andige Programme, welche innerhalb des ausf"uhrenden Rechners (Client) gestartet werden. Der Client fungiert hier gleichzeitig als Server. Java-Applikationen haben die M"oglichkeit auf das Dateisystem des Clients, bzw. Server zuzugreifen.
Z.B. kann die Java-Applikation Foo mittels 

\begin{verbatim}
java Foo
\end{verbatim}

aufgerufen werden. Hierf"ur werden lediglich die JVM und die entsprechenden \emph{.class}-Files ben"otigt.

Ein Java-Applet wird auf einem Server abgelegt. Das Applet wird auf dem Client "ubertragen und dort lokal ausgef"uhrt. Im Gegensatz zur Java-Applikation kann das Applet nicht auf Daten ausserhalb der JVM zugreifen. Somit kann ein Applet die Sicherheit des Systems nicht gef"ahrden.
Java-Applets werden innerhalb einer HTML\footnote{Hypertext Markup Language, Textformat f"ur World-Wide-Web(WWW) Seiten}-Seite aufgerufen.

Das Applet wird mittels des Internet, bzw. Intranet, auf den Client "ubetragen.
Hierf"ur kann ein Java-F"ahiger-WWW-Browser oder der \emph{appletviewer} des JDK benutzt werden. Z.B. kann das Java-Applet Foo eingebettet in der HTML-Seite \emph{Foo.html} mittels 
\begin{verbatim}
appletviewer http://www.server.de/Foo.html
\end{verbatim}
aufgerufen werden.
Mittlerweile haben die meisten Unternehmen Java lizensiert 
und in ihren Web-Browsern unterst"utzt.
Dadurch lassen sich Web-Seiten interaktiver gestalten.


\section{Objekt-Orientierte-Sprache}

\subsection{Was ist Objektorientierung?}
Java ist eine reine Objekt Orientierte Programmiersprache.
Im Gegensatz zu einer prozeduralen Programmiersprache, die sich durch einen
linearen Programmflu"s und einer klaren Trennung von Daten und 
Funktionen auszeichnet verkn"upft das objektorientierte Programmieren Daten und 
Methoden\footnote{Methoden ist der objektorientierte Name f"ur Funktionen} 
zu einem Objekt.
Objekte eines Typs lassen sich zu einer Objektklasse zusammenfassen.
Klassen sind Beschreibungen eines Datenmodells (Prototypen)
von denen sich beliebig viele Objekte generieren lassen.
In den Klassen werden sowohl die Daten als auch die damit verbundenen 
Operationen festgelegt.

\subsection{Klassen und Methoden}
In Java wird eine Klasse mittels des Schl"usselwortes \emph{class} (Analog zu
Klassen in C++) aufgerufen.

Der Zugriff auf Daten und Methoden k"onnen innerhalb einer Klasse 
f"ur andere Klassen mittels \emph{public}-deklaration erlaubt, bzw. 
mit \emph{private}- oder \emph{protected}-deklaration verboten werden.

Z.B. definiert die Klasse Feuerzeug Daten und Methoden f"ur Objekte des selben Typs.

\begin{verbatim}
public class Feuerzeug 
{ 
	/* Gasinhalt in Liter */
	private float gasInhalt; 

        /* 1 milliliter pro Zuendung */	
	private float finalize gasProZuendung = 0.001 

	/* Oeffentliche Methode fuer das Zuenden des Feuerzeugs.
           Beachte: Das Zuenden kostet Gas !
	*/
	public boolean zuende()
	{
		if(gasInhalt<gasProZuendung)
			return false;
		gasInhalt-=gasProZuendung;
		return true;
	}
}
\end{verbatim}


Eine Methode ist unmittelbar an ein Objekt bzw. die dazugeh"origen Daten
gebunden. Methoden k"onnen in Java auch "uberladen werden.
Z.B. 

\begin{verbatim}
public class Feuerzeug 
{ 
	...

	// Feuerzeug ganz aufladen
	void aufLaden()
	{
		...
	}

	// Feuerzeug mit definierter Menge 'vol' aufladen
	void aufLaden( float vol )
	{
		...
	}

}
\end{verbatim}


Durch Methoden werden bestimmte Funktionen eines Objektes beschrieben.
Sie werden in Java mittels des \emph{.} Operators aufgerufen.

Objekt.Methode (....)

Konstruktoren sind spezielle Methoden, die bei der Aktivierung von Objekten 
aufgerufen werden und deren Instanzierung und Initialisierung "ubernehmen.
Konstruktoren benutzen den gleichen Namen wie die verwendete Klasse.

Klasse Objekt=new KLasse (...)

erzeugt aus der Klasse ein Objekt, daf"ur reserviert es 
Speicher f"ur dieses Objekt
und initialisiert es mit den entsprechenden Werten ( evt. als Argumente "ubergeben).

\subsection{Vererbung}

Vererbung bezeichnet den Mechanismus des Ableitens einer Klasse aus einer 
"ubergeordneten Klasse (Einfachvererbung).
Die Daten und Methoden der "ubergeordneten Klasse, 
Oberklasse oder Superklasse genannt,
werden an die Subklasse weitergegeben (vererbt).
Die vererbten Methoden k"onnen dort auch "uberschrieben, bzw. "uberladen
werden.

Die Klassen der standard Bibliothek sind in der Regel alle von der superklasse
\emph{Object} abgeleitet. Die Klasse \emph{Object} definiert einige elementare Methoden (equals, string, getclass), welche von allen Subklassen benutzt, bzw. neu definiert werden k"onnen.

Im Gegensatz zu C++ k"onnen in Java keine Operatoren "uberladen, bzw. neu definiert werden. Als L"osung werden in Java standard Methoden verwendet, wie z.B. die Methode \emph{equals} der Superklasse \emph{Object}.

\subsection{Packages}

Ein wesentlicher Vorteil der objektorientierten Programmierung ist die
einfache Wiederverwendung bestehenden Programmcodes.
Klassenbibliotheken bieten Programmierern die m"oglichkeit 
bestehende Klassen in Ihrer Applikation einzubinden.
Es existieren eine Reihe von standard Klassenbibliotheken (Packages) f"ur 
unterschiedliche Zwecke wie z.B. Grafik, Netzwerkaktivit"aten und Ein-und
Ausgabefunktionen.
Packages werden mit dem \emph{import}-Befehl in den Programmcode eingebunden.
Die Zusammenfassung von mehreren Klassen in Packages erleichtert 
die Realisierung und Modularisierung gro"ser Systeme.
Packages dienen ebenfalls der einfachen Wiederverwendung.

\section{System-Managment}

\subsection{Garbage-Collector}
Ein in Java integrierter Garbage-Collector sorgt f"ur die automatische
Freigabe nicht mehr ben"otigter Speicherbereiche.
Dadurch werden potentielle Fehler vermieden, wie sie z.B. bei doppelter Speicherfreigabe in ANSI-C vorkommen k"onnen.
W"ahrend der Laufzeit eines Java-Programmes arbeitet im Hintergrund der 
Garbage Collector als Thread niedriger Priorit"at.

\subsection{Exceptions}
Mit den Exceptions besitzt Java einen Mechanismus zur strukturierten 
Behandlung von Fehlern, die w"ahrend der Programmausf"uhrung auftreten.
Tritt ein Laufzeitfehler (z.B. ein Array-Zugriff au"serhalb der Array-Grenzen), 
so wird ein eine entsprechende Exception generiert (\emph{throw}).
Diese Exception kann, bzw. mu"s innerhalb eines \emph{try-catch}-Blocks 
abgefangen werden (\emph{catch}). 

Das Grundprinzip des Exception-Mechanismus in Java kann wie folgt beschrieben
werden:
Wird eine Exception ausgel"ost, arbeitet die JVM den Stack soweit zur"uck,
bis sie eine entsprechende \emph{catch}-Anweisung findet. Wird keine geeignete gefunden, gibt die JVM eine Fehlermeldung aus.

Eine Exception kann an jeder Stelle der Aufrufkette (Stack) abgefangen werden.
Methoden, welche in der Deklaration angeben eine Exception ausl"osen zu k"onnen, m"ussen innerhalb eines \emph{try-catch} Blocks aufgerufen werden.

Das Behandeln von Exceptions mit der try-catch-Anweisung sieht etwa
folgenderma"sen aus :

\begin{verbatim}
try {
    Anweisung;  
     
    ...
    }
catch ( Exceptiontyp1 x ) {
    Anweisung;  

   ...
   }
catch ( Exceptiontyp2 x ) {
    Anweisung;  

   ...
   }
\end{verbatim}

\section{Grafisches User Interface (GUI)}

\subsection{Was ist eine GUI?}
AWT
Swing ..

\subsection{Abstract Window Toolkit (AWT)}

\subsection{Events}

Benutzer-Interaktion !

Bei der Programmierung unter einer grafischen Oberfl"ache erfolgt die
Kommunikation zwischen Betriebssystem und Anwendungsprogramm durch den 
Austausch von Nachrichten.
Die Anwendung wird dabei "uber alle Arten von Ereignissen und 
Zustands"anderungen vom Betriebssystem informiert.
Dazu z"ahlen z.B. Mausklicks, Bewegung des Mauszeigers, oder Ver"anderungen 
der Fenstergr"o"se oder dessen Lage.
Jedes dieser Ereignisse oder \emph{ Events} l"ost eine Nachricht aus, die an das 
aktive Fenster gesendet wird. Dadurch wird dort der Aufruf einer \emph{Callback}-Methode
ausgel"ost. Das Programm kann auf ein \emph{Even}t reagieren, indem es die zugeh"orige
Methode "uberlagert und mit der gew"unschten Funktionalit"at ausstattet.
Die unter Java m"oglichen Events lassen sich grob in folgende Klassen
unterteilen :
\begin{itemize}
\item Maus-Events
\item Tastatur-Events
\item Fenster-Events
\item Action-Events
\item Komponenten-Events
\end{itemize}

Z.B. wird ein Mausklick mit den Methoden \emph{mouseDown} und \emph{mouseUp} der 
Klasse \emph{Component} behendelt.
Wenn die Maustaste gedr"uckt wird, ruft das AWT die Methode \emph{mouseDown} auf,
l"asst man sie los wird die Methode \emph{mouseUp} aufgerufen.

public boolean mouseDown(Event e, int x, int y);
public boolean mouseUp(Event e. int x, int y);

Der erste Parameter beider Methoden ist eine Instanz Klasse \emph{Event}.
Die beiden anderen Parameter geben die Position in der Client-Area an,
an der die Maustaste gedr"uckt bzw. losgelassen wurde.
Ein Objekt der Klasse \emph{Event} repr"asentiert das Ereignis, durch das die
Nachricht ausgel"ost wurde.
Ein \emph{Event}-Objekt besitzt eine Reihe "offentlicher Instanzmerkmale
wie z.B.:


\begin{tabular}{|l|l|}
\hline
Element & Bedeutung \\
\hline
public int x; &	x-Koordinate des Mauszeigers bei Tastatur- \\
              & und Mausereignissen \\
\hline
public int y; &	y-Koordinate des Mauszeigers bei Tastatur- \\
              & und Mausereignissen \\
\hline
public long when; & Zeitpunkt des Ereignisses \\
\hline
public int key; & ASCII-Wert der gedr�ckten Taste bei \\
                & Tastaturereignissen bzw. Wert der Funktionstaste \\
\hline
public int modifiers; &	Eine Kombination der Konstanten \\
                      & \emph{Event.ALT\_MASK, Event.CTRL\_MASK,} \\
                      & \emph{Event.META\_MASK, Event.SHIFT\_MASK} \\
\hline
\end{tabular}


\section{Multithreading}

\subsection{Was ist Multithreading?}

Durch die Weiterentwicklungen im Bereich der Betriebssystemtechnologie
wurde das Konzept der \emph{Threads} in den letzten Jahren immer popul"arer.
Und durch bereitstellung der Basis von Library-Routinen auch konventionellen 
Programmiersprachen zur Verf"ugung gestellt.
Java hat \emph{Threads} direkt in die Sprache integriert und mit den erforderlichen
Hilfsmitteln als Konstrukt zur Nebenl"aufigkeit implementiert.
Ein \emph{Thread} ist ein eigenst"andiges Programmierfragment, das parallel
zu anderen \emph{Threads} laufen kann.
Der Laufzeit-Overhead zur Erzeugung und Verwaltung ist deutlich geringer
als bei gew"ohnlichen Prozessen und kann in den meisten Programmen 
vernachl"assigt werden.
\emph{Threads} sollen unter anderem die Implementierung grafischer Anwendungen
erleichtern, die durch Simulation komplexer Abl"aufe oft nebenl"aufig sind.
\emph{Threads} k"onnen auch dazu verwendet werden, die Bedienbarkeit von 
Dialoganwendungen zu verbessern, indem rechenintensive Anwendungen
im Hintergrund ablaufen.

\subsection{Programmierung von Threads}

\emph{Threads} werden in Java durch die Klasse\emph{ Thread} und das Interface \emph{Runnable}
implementiert. in beiden F"allen wird der \emph{Thread-Body}, also der
parallel auszuf"uhrende Code, in Form der "uberlagerten Methode \emph{run} zur
Verf"ugung gestellt. Die Kommunikation kann dann durch Zugriff auf die 
Instanz- oder Klassenvariablen oder durch Aufruf beliebiger Methoden,
die innerhalb von \emph{run} sichtbar sind, erfolgen.
Zur Synchronisation stellt Java das aus der Betriebssystemtheorie bekannte
Konzept des \emph{Monitors} zur Verf"ugung. Dises  erlaubtes, kritische Abschnitte
innerhalb korrekt geklammerter Programmfragmente und Methoden zu kapseln.
Somit wird der Zugriff auf gemeinsam benutzte Datenstrukturen koordiniert.
Dar"uber hinaus stellt Java Funktionen zur Verwaltung von \emph{Threads}.
Diese erlauben es, \emph{Threads} in Gruppen zusammenzufassen, zu priorisieren
und Informationen "uber Eigenschaften von \emph{Threads} zu gewinnen.
Das \emph{Schedulling} kann dabei wahlweise unterbrechend oder nichtunterbrechend
implementiert sein.
Die Sprachspezifikation legt dies nicht fest, aber in den meisten 
Java-Implementierungen wird dies von den M"oglichkeiten des darunter liegenden
Betriebssystems abh"angen.

Die Klasse \emph{Thread} ist Bestandteil des Packages \emph{java.lang} und steht damit
allen Anwendungen standartm"a"sig zur Verf"ugung.
\emph{Thread} stellt die Basismethoden zur Erzeugung, Kontrolle und zum Beenden von 
\emph{Threads} zur Verf"ugung. Um ein konkreten \emph{Thread} zu erzeugen, mu"s eine
eigene Klasse aus \emph{Thread} abgeleitet und die Methode \emph{run} "uberlagert werden.
Mit Hilfe des Aufrufs der Methode \emph{start} wird der \emph{Thread} gestartet und die weitere
Ausf"uhrung an die Methode \emph{run} "ubertragen.
\emph{start}wird nach dem starten des \emph{Threads} beendet, und der Aufrufer kann paralell
zum neu erzeugten \emph{ Thread} fortfahren.
Die "ubliche Vorgehensweise, einen \emph{Thread} zu beenden, besteht darin, die Methode \emph{stop}
der Klasse \emph{Thread} aufzurufen.
In diesem Fall wird der \emph{Thread} angehalten und aus der Liste der aktiven \emph{Threads}entfernt.
Mit Hilfe der Methoden \emph{suspend} und \emph{resume} ist es m"oglich, einen \emph{Thread} vor"ubergehend
 zu unterbrechen. Durch \emph{suspend} wird die Ausf"uhrug unterbrochen, und durch \emph{resume}
wird der \emph{Thread} (genauer gesagt: die Methode \emph{run})an der Stelle fortgesetzt, an der die
Unterbrechung erfolgte.


\section{Java Native Interface (JNI)}

\subsection{Was ist JNI?}
Java stellt mit den sogenannten \emph{nativen} Methoden die M"oglichkeit,
nicht in Java geschriebene Programmes oder Abl"aufe in die Laufzeitbibliothek
 mit einzubinden.
Gr"unde f�r die Implementierung von \emph{nativen} Methoden sind folgende:
\begin{itemize}
\item das man spezielle F"ahigkeiten der einzelnen Rechner oder der 
Betriebssysteme nutzen kann; die die Java-Bibliothek nicht bereitstellt.
Dazu geh"ort z.B. der Anschlu"s an neue Peripherieger"ate oder Steckkarten,
der Zugriff auf verschiedene Netztypen oder die Verwendung eines eindeutigen
Merkmals des vorhandenen Betriebssystems.
Solche F"ahigkeiten werden derzeit von der Java-Umgebung nicht bereitgestellt
und m"ussen daher \emph{au"serhalb} von Java in einer anderen Sprache (meist 
C oder eine mit C vertr"aglichen Sprache) implementiert werden.
\item Ein weiterer Grund ist die Performance des auszuf"uhrenden Programms.
Falls man den von Java zur Verf"ugung gestellten JIT-Compiler f"ur die 
Erh"ohung der Geschwindigkeit nicht einsetzen will, kann man f"ur den 
Geschwindigkeitsorientierten Teil ( z.B. kritische innere Schleifen ),
in C geschriebene \emph{native} Methoden einsetzen.
Die Java-Klassenbibliothek nutzt diese M"oglichkeit bei bestimmten 
kritischen Systemklassen selbst, um die Effiziens des Systems zu steigern.
\item Jedoch ist wohl der Hauptgrund f"ur den Einsatz von nativen Methoden in 
Java-Klassen der; da"s man auf diese Art und Weise schon existierende Programme
ohne sie erst umzuschreiben, relativ einfach in Java einbinden kann.
Diese M"oglichkeit haben wir in der Diplomarbeit auch genutzt.
Die 3D-OpenGL-Animationen sind in C geschrieben und wurden sozusagen
in Java eingebettet, doch dazu sp"ater mehr.
\end{itemize}

\subsection{Grundlagen der JNI}

Dynamische Bibliothek
	Unix: lib<Name>.so, LD\_LIBRARY\_PATH
	WIN: <Name>.dll, SYSTEMVERZEICHNIS

\subsection{Vorgehensweise}

Um ein Interface zwischen Java und dem C-Code zu erstellen, sind im wesenlichen
f"unf Schritte erforderlich :
\begin{enumerate}
	\item Erstellen der Java-JNI Funktionsdeklaration
	\item Generierung der C-Header Datei
	\item Erstellen der C-JNI Functionen
	\item Laden der JNI-Bibliothek in Java
	\item Linken der Dynamischen Bibliothek
\end{enumerate}

\subsection{Erstellen der Java-JNI Funktionsdeklaration}
Den Java-Code mit der deklarierten \emph{nativen} Methode zu schreiben und zu 
"Ubersetzen.

Um \emph{native} Methoden in Java-Klassen zu benutzen, mu"s man in der Deklaration
der Methode lediglich das schl"usselwort \emph{native} einf"ugen.

wie z.B. 
\begin{verbatim}
public nativ void Methodenname();
\end{verbatim}

Und die Methode System.loadLibrary() in die Klasse mit einbinden.
Das Schl"usselwort \emph{native} zeigt dem javac-Compiler und dem java-Interpreter
das sie nach einem Methodenrumpf in einer dynamisch-ladbaren Bibliothek
Ausschau halten m"ussen.
Um die gew"unschte Bibliothek letzendlich wirklich aufzurufen, wird sie mit der
Methode \emph{System.loadLibrary()} aus dem System-Package von Java eingelesen.
Java durchsucht alle Pfade die in dem Betriebssystem-Umgebungsvariable
angegeben wurden.
Alternativ steht die Methode\emph{ Runtime.getRuntime().loadLibrary()} und die 
Methode \emph{System.load()} zur Verf"ugung.
Letztrige liest eine Klassen-Bibliothek anhand ihres vollst"andigen 
Pfadnamens ein. So k"onnen auch Bibliotheken au"serhalb des Suchpfades 
benutzt werden.
Die "Ubersetzung des Java-Codes kann wie "ublich ausgef"uhrt werden.

\subsection{Generierung der C-Header Datei}
Der n"achste Schritt ist es, die "ubersetzte Datei \emph{Datei.class} dazu zu benutzen, um die
entsprechende Header-Datei \emph{Datei.h} zu erzeugen.
Dazu wird das \emph{javah}-Tool aus der JDK-Distribution eingesetzt.
Per Default wird die neue h-Datei im gleichen Verzeichnis angelegt. Mit der 
Option -d kann ein anderes Verzeichnis angegeben werden. 
Dies ist zu raten da noch diverse andere Dateien erzeugt werden m"ussen und
schnell die "Ubersicht verloren gehen kann.
Der Aufruf 

\begin{verbatim}
javah -jni Datei
\end{verbatim}

erzeugt das C-Headerfile \emph{Datei.h}. 

In der Header-Datei wird durch die typedef-struct-Anweisung der \emph{nativen}
C-Routine mitgeteilt, wie die Daten in der Java-Klasse angeordnet sind. Die einzelnen Variablen
in der Struktur k"onnen benutzt werden, um die Klassen
und Instanz-Variablen von Java zu benutzen.
Weiterhin wird in dem h.File ein Prototype angegeben, um die Methode aus dem 
objektorientierten Namensraum der Java Klasse in den C-Namensraum zu 
"uberf"uhren.
Der Name einer Funktion, der eine \emph{native} Methode implementiert, ergibt sich 
dabei immer aus dem Packet-Namen, dem Klassen-Namen und dem Namen der nativen
Methode von Java, getrennt durch einen Unterstrich.

\subsection{Erstellen der C-JNI Functionen}
Nun m"ussen die nativen Methoden implementiert werden.
"Ublicherweise werden die Implementationsdateien mit dem Klassennamen und einer
eindeutigen Endung z.B. \emph{Datei.c} versehen.
Die Implementationsdatei mu"s die Datei \emph{jni.h} mittels \#include-Anweisung einbinden. \emph{jni.h} ist im \emph{include} Verzeichniss des JDK enthalten.
Ebenfalls muss die mittels \emph{javah -jni} erzeugte Headerdatei eingebunden werden. 
Der Funktionskopf in der Implementationsdatei muss den 
generierten Prototypen aus der Headerdatei entsprechen.

Diese Fehlerart macht sich noch nicht beim Linken sondern erst zur Laufzeit 
des Programms bemerkbar und ist daher m"uhsam zu beheben.

\subsection{Erstellung der Dynamischen Bibliothek}

Nachdem nun alle ben"otigten Dateien vorhanden sind, mu"s die 
Implemtierungsdatei nur noch "ubersetzt und mit der Java-Bibliothek  
zu einer dynamischen Bibliothek zusammen gelinkt werden.
  
\subsection{Datenaustausch zwischen Java und C}

Jedem elementaren Typ in Java wird ein Typ in der Prototype-Funktion
und damit auch in C zugeordnet
Ein char in Java ist nach dem UNICODE kodiert, im Gegensatz zu ASCII in C.
Dem Java-Typ String kann nur eine Struktur zugeordnet werden. 
Funktionen zur Umwandlung von UNICODE und ASCII sowie die Definition
String-Strukturen finden sich in der Header-Datei wieder javaString.h .
 
Variablen nicht als Parameter einer Methode sondern als Klassenvariable
anzulegen ist objektorientiert und erm"oglicht einen einfachen Zugriff
von C aus. Dieser Zugriff kann als 'Call by Reference' bezeichnet werden.
"Ubergibt man die Variablen als Parameter ist der Zugriff ein
Call by Value.
Alle Klassenvariablen werden im .h-File zu einem \emph{struct} zusammengefa"st.
Auf diesen erh"alt die aufgerufene C-Funktion als Parameter ein
Handle-Pointer.
Das in \emph{StubPreamble.h} definierte, \emph{unhand()}-Makro erm"oglicht den Zugriff
auf die einzelnen Klassenvariablen. Das \emph{unhand()}-Makro "ubernimmt einen
Pointer auf das Handle einer Klasse und gibt einen Pointer auf die im 
.h-File erzeugte Klassenstruktur zur"uck.
"Uber den R"uckgabewert des Makros lassen sich die Instance-Variablen
der Java-Klasse direkt auswerten und ver"andern.

\subsubsection{Strings}
Wie schon erw"ahnt bilden \emph{Strings} in Java eine eigene Klasse.
M"ochte man einen \emph{String} von Java nach C oder umgekehrt "ubergeben,
mu"s man \emph{javaString.h} im C-Implementationsfile mit einbinden.
In \emph{javaString.h} finden sich die Typdefintionen und Funktionen um \emph{Strings}
von Java nach C und umgekehrt zu transformieren.
So gibt es noch weitere Methoden um \emph{Strings} zu bearbeiten, wie z.B.
\emph{MoveString\_GetString(), CString()} oder \emph{makeJavaString()}.
Java-\emph{Strings} bilden eine eigene Klasse, somit wird bei der Konvertierung nicht 
nur ein elementarer Typ sondern eine ganze Klasse an die \emph{native} Methode
durchgereicht. Dieses Konzept l"a"st sich auch auf andere Klassen erweitern.

\subsubsection{Felder}
\emph{Felder} werden von Java nach C "ubergeben indem vom \emph{javah}-Tool ein 
Funktionsparameter von Typ \emph{struct HArrayOf<Objekt> *} erzeugt wird.
Dieser Handle enth"alt ein Element \emph{body} mit dem auf die Feldelemente 
zugegriffen werden kann.
Der Zugriff auf das Feld erfolgt mit dem \emph{unhand()}-Makro und dem
\emph{body}-Element.

\subsubsection{Speichermanagment}
in Java wird die Speicherverwaltung automatisch vom \emph{Garbage-Collector} 
erledigt indem er die Methode \emph{dispose()} aus der Java-Klasse ausf"uhrt.
Wird in einer C-Funktion einer \emph{nativen} Methode Speicher angelegt,
so h"angt die Vorgehensweise des Programmierers beim Freigeben,
von der Art des Speichers ab.
Bei normalen Speicherbereichen f"ur die interne C-Anwendung mu"s der 
Programmierer sich explizit um die Freigabe k"ummern. 
Da der Java-Interpreter von diesem Speicher nichts wei"s,
m"ussen eigene \emph{native} Methoden implementiert werden um ihn freizugeben.
In C erzeugte Speicherbereiche von Java-Objekten werden nicht unbedingt
vom \emph{Garbage-Collector} gefunden. Diese Java-Objekte, also Instanzen 
einer bestimmten Java-Klasse, k"onnen auch aus Java heraus freigegeben 
werden. Eine einfache Variante der Freigabe ist der explizite Aufruf
der Methode \emph{dispose()} wenn die Instanz nicht mehr ben"otigt wird.
%
%\begin{thebibliography}{99}
%	\bibitem{1} Mary Campione and Kathy Walrath; The Java Tutorial; Sun Microsystems; http://www.javasoft.com/docs/books/tutorial/index.html
%	\bibitem{2} Ron Fosner; OpenGL Programming for Windows 95 and Windows NT; Addison Wesley Second Printing April 1997; http://www.aw.com/devpress
%	\bibitem{3} Woo, Mason; OpenGL programming guide 2nd ed., Addison Wesley Developer Press April 1997, ISBN 0-0201-46138-2; http://www.aw.com/devpress
%\end{thebibliography}
%
%\end{document}
% 
}

\begin{thebibliography}{99}
	\bibitem{SGI97} Silicon Graphics: OpenGL; \htmladdnormallink{http://www.opengl.org}{http://www.opengl.org}

	\bibitem{Woo97} Woo, Mason: OpenGL Programming Guide 2nd Edition; Addison Wesley Developer Press April 1997; ISBN 0-0201-46138-2; \htmladdnormallink{http://www.aw.com/devpress}{http://www.aw.com/devpress}

	\bibitem{Fosner97} Ron Fosner: OpenGL Programming for Windows 95 and Windows NT; Addison Wesley Second Printing April 1997; \htmladdnormallink{http://www.aw.com/devpress}{http://www.aw.com/devpress}
	
	\bibitem{Barkakati97} Nabajyoti Barkakati: X Window System Programming; Sams Publishing Secound Edition

	\bibitem{JavaTuto} Mary Campione and Kathy Walrath: The Java Tutorial; Sun Microsystems; \htmladdnormallink{http://www.javasoft.com/docs/books/tutorial/index.html}{http://www.javasoft.com/docs/books/tutorial/index.html}

	\bibitem{Chan97} Leo Chan (lchan@cgl.uwaterloo.ca): OpenGL4Java; \htmladdnormallink{http://ftp.cgl.uwaterloo.ca/pub/software/meta/OpenGL4java.html}{http://ftp.cgl.uwaterloo.ca/pub/software/meta/OpenGL4java.html}

	\bibitem{King97} Adam King (adam@opcom.ca): OpenGL4Java; \htmladdnormallink{http://www.magma.ca/\%7Eaking/java}{http://www.magma.ca/\%7Eaking/java}

	\bibitem{Reilly97} Tommy Reilly (tom@pajato.com): Jogl; \htmladdnormallink{http://www.pajato.com/jogl}{http://www.pajato.com/jogl}

	\bibitem{JavaCC} Sun Microsystems: JavaCC Version 1.1; \htmladdnormallink{http://www.metamata.com/JavaCC/}{http://www.metamata.com/JavaCC/}

	\bibitem{Kopka} Helmut Kopka: \LaTeX\ : Eine Einf"uhrung; Addison-Wesley 1991; ISBN 3-89319-338-3 

	\bibitem{GnuWin32} Cygnus Solutions, GNU-Win32 Project Version b18; \htmladdnormallink{http://www.cygnus.com/misc/gnu-win32}{http://www.cygnus.com/misc/gnu-win32}

	\bibitem{ProfBunse} Prof. Dr. Wolfgang Bunse (bunse@fhzinfo.fh-bielefeld.de); Fachhochschule Bielefeld; \htmladdnormallink{http://www.fh-bielefeld.de}{http://www.fh-bielefeld.de}

	\bibitem{GL4JavaHome} Goethel: GL4Java Homepage; \htmladdnormallink{http://www.jausoft.com}{http://www.jausoft.com}

	\bibitem{Dragos96} Nikos Drakos (nikos@cbl.leeds.ac.uk): Latex2Html; \htmladdnormallink{http://cbl.leeds.ac.uk/nikos/tex2html/doc/latex2html/latex2html.html}{http://cbl.leeds.ac.uk/nikos/tex2html/doc/latex2html/latex2html.html}

\end{thebibliography}

\end{document}
