%\documentclass[a4paper]{report}
%\usepackage{german}

%
%\begin{document}
%
%\begin{titlepage}
%  {\LARGE
%      \begin{tabular}{p{20mm}l}
%	      \multicolumn{2}{r}{\bf FH--Bielefeld}\\[4mm]
%	      \hline \\[3mm]
%	      & Einf"uhrung\\
%	      & f"ur\\
%	      & OpenGL \& Java\\[5mm]
%	      & Atilla Kolac\\[5mm]
%	      & Labor f"ur Parallelverarbeitung \\
%	      & FH-Bielefeld\\[5mm]
%	      \hline \\[20mm]
%	      & \today
%      \end{tabular}
%      }
%\end{titlepage}
%
%\tableofcontents
%

\chapter{OpenGL Einf"uhrung\ddag}

\section{Was ist OpenGL?}
OpenGL \cite{Woo97} \cite{Fosner97} wurde von Silicon Graphics,Inc.(SGI) entwickelt und ist eine verbreitete
Hardware- und Betriebsunabh"angige Bibliothek f"ur die Erstellung von 
dreidimensionalen Grafiken.
GL steht hierbei f"ur Grafics Library.

Au"serdem stellt OpenGL eine unabh"angige Programmierschnittstelle dar.
Wobei f"ur die Hardwareseite eine geeignete Implementation programmiert werden
kann. Die Softwareschnittstelle besteht aus ca. 120 Kommandos, die zur
Spezifikation von Objekten und Operationen ben"otigt werden.
OpenGL ist client- und serverf"ahig. Das Protokoll welches OpenGL-Befehle 
"ubermittelt, ist identisch. Dadurch k"onnen OpenGL-Programme
im Netzwerk ablaufen, in denen client und server Rechner verschiedenen
Typs sind. Dies erreicht man indem die Kommandos zum Ablauf von Windows-Tasks
oder zur Interaktion mit dem Benutzer nicht in dem OpenGL-Programm implementiert
werden. Sondern diese Aufgaben werden direkt mit der 
Fensteroberfl"achenprogrammierung der
entsprechenden Hardware mit"ubernommen.

Der Sinn der OpenGL-Bibliothek ist es, zwei- und dreidimensionale Objekte in 
einem Frame Buffer zu \emph{rendern}. Dieser ist mit einem Pixelspeicher in der Grafik-
Hardware eines PC vergleichbar. 
OpenGL ist streng prozedural. Das hei"st,
da"s nicht das Aussehen eines Objektes beschrieben wird sondern wie das 
Objekt gezeichnet wird. Hierzu bedient sich OpenGL zwei - oder dreidimensionalen
\emph{Vertices}. Diese repr"asentieren einen Punkt, z.B. den Endpunkt einer Linie 
oder eines Polygons. Die n"achste Ebene sind \emph{Primitive}, die aus einem oder
mehreren \emph{Vertices} bestehen.

Wie Vertices zu Primitiven zusammengesetzt werden und in ein Frame Buffer 
gezeichnet werden, wird durch eine Vielzahl von Einstellm"oglichkeiten 
kontrolliert.

Mit OpenGL kann man weiterhin Oberfl"achen zeichnen, Beleuchtungsspezifikationen
anwenden und Texturen verwenden.

Die wichtigsten Schritte bis zum Rendern einer Szene

\begin{enumerate}

\item Aus den geometrischen Primitiven werden Formen konstruiert und damit
mathematische Beschreibungen der Objekte erstellt. 
OpenGL sieht Punkte, Linien, Polygone, Bilder und Bitmaps als Primitive an.

\item Die Objekte werden im dreidimensionalen Raum arrangiert und der gew"unschte 
\emph{Blickpunkt} "uber der Szene wird ausgesucht.

\item Die Farben aller Objekte werden berechnet.
Zwar sind die Farben f"ur die Objekte im Programm definiert,
m"ussen jedoch neu berechnet werden falls sich das Objekt bewegt oder die 
Schattierung sich "andert.

\item Die mathematische Beschreibung von Objekten und den ihnen beigef"ugten
Farbinformationen werden in Bildschiermpixel konvertiert.
Dieser Proze"s wird \emph{Rasterung} genannt.

\end{enumerate}


Alle Daten, ob sie nun Geometrien oder Pixel beschreiben, k"onnen in einer
\emph{Display List} gesichert werden.
OpenGL unterscheidet zwischen zwei Berechnungsformen: 
Die Berechnung der Grafik wird sofort ausgef"uhrt (\emph{immediate-mode}).
Das bedeutet wenn das Kommando zum Zeichnen eines Objektes abgesetzt wird,
wird das Objekt gezeichnet. Dieser Modus ist in OpenGL voreingestellt.
Weiterhin lassen sich Kommandos in einer \emph{Display List} sichern, um sie sp"ater
auszuf"uhren.
Die sofortige Ausf"uhrung der Grafik l"asst sich einfacher programmieren,
das Sichern in einer Liste ist effizienter.

\section{Grundlagen}

\subsection{Client- Serverprotokoll}

\subsection{Datentypen}

\subsection{Beleuchtung}
OpenGL berechnet die Farben jedes Pixels, bevor die Szene dargestellt wird
und speichert sie im Frame Buffer ab. Dabei wird das in der Szene benutzte Licht
berechnet, und die Art, wie die Objekte in der Szene das Licht reflektieren 
oder absorbieren sollen.
Ein unbeleuchtetes dreidimensionales Objekt kann nicht von einem 
zweidimensionalen Objekt unterschieden werden.
Denn auf einer zweidimensionalen Fl"ache, sprich unser Bildschirm, 
wird die dreidimensionalit"at durch die Art der Beleuchtung und der
Schattierung erreicht.
Das zeigt, wie wesentlich die Beziehung zwischen Objekten und dem Licht 
in einer dreidimensionalen Szene ist.
OpenGL stellt Licht und Schatten durch Zerlegung der Objektoberfl"ach in
bestimmte rote, gr"une und blaue Komponenten dar.
Der Ort der Lichtquelle, sein Einstrahlungswinkel und Art m"ussen programmiert
werden.

Das Beleuchtungsmodell in OpenGL besteht aus vier Komponenten:
\emph{emitted, ambient, diffuse und specular}.
Alle vier Komponenten bewirken eine unterschiedliche Beleuchtungsart.
Sie werden getrennt angegeben und danach zusammenaddiert,
so das die Szene in das gew"unschte Licht gesetzt wird.

\section{Die Programmierung}

\subsection{Namenskonvention}

\subsection{Die Kommandosyntax}
OpenGL-Kommandos beginnen in der Programmiersprache C 
mit dem Pr"afix \emph{gl}, 
jedes Wort beginnt mit einem Gro"sbuchstaben ( z.B. glClearColor() ).
Konstanten beginnen mit dem gleichen Pr"afix, 
allerdings werden sie in Gro"sbuchstaben geschrieben und durch Unterstriche 
verbunden ( z.B. GL\_CLEAR\_BUFFER\_BIT ).
in dem Kommando \emph{glColor3f()} steht 3 f"ur die Anzahl der gegebenen Argumente,
f steht f"ur die Gleitkommazahlen.

Die ANSI-C-Implementation von OpenGL erm"oglicht bis zu 8 verschiedene
Datentypen.
Die Datentypen umfassen 8, 16 und 32-Bit-Integer, im signed- und unsigned-
Format und 32- oder 64-Bit-Gleitkommazahlen.
Einige OpenGL-Kommandos sind mit dem Endbuchstaben v versehen, was darauf hindeutet, da"s
das Kommando einen Zeiger auf einen Vektor oder Feld beinhaltet.

\subsection{Die Bibliotheken}
OpenGL bietet eine Anzahl leistungsf"ahiger und einfacher Kommandos zum Rendern.
Alle komplexen Zeichnungen m"ussen mit diesen Kommandos ausgef"uhrt werden.
Deshalb ist es m"oglich, eine eigene Bibliothek zu schreiben, die auf 
OpenGL aufsetzt. Damit l"a"st sich z.B. die Behandlung der Fenster vereinfachen.

Hier sind zwei sehr verbreite Bibliotheken aufgelistet:
\begin{itemize}
\item Die \emph{OpenGL Utility Library (GLU)} enth"alt einige Routinen die einfachere
OpenGL Kommandos benutzt. Es werden Funktionen zur Vereinbarung von Matrizen f"ur
besondere Betrachtungsorientierungen und Projektionen und zum Rendern von Oberfl"achen
definiert.
GLU-Funktionen sind durch den Pr"afix \emph{glu} zu erkennen.

\item Die OpenGL-Erweiterung f"ur das X-Windows System (GLX)  sieht die Erzeugung 
eines OpenGL-Kontextes vor und die Hinzuf"ugung eines Fensters, in dem gezeichnet
werden kann (drawable window), falls auf einem Rechner mit X-Window System 
gearbeitet wird.
GLX-Funktionen besitzen das Pr"afix \emph{glx}.
\end{itemize}
 
Hinzukommen noch eine gro"se Anzahl weiterer Bibliotheken, die wir hier 
verst"andlicherweise nicht aufz"ahlen k"onnen.


 
\subsection{Die OpenGL Windows-Programmierung}
Bevor die OpenGL-Bibliothek unter MS-Windows verwendet werden kann, m"ussen 
eine Reihe von Initialisierungsschritte ausgef"uhrt werden.
Jede Windows-OpenGL-Applikation mu"s seinen \emph{Rendering Context} ( OpenGL-Seite )
mit seinem \emph{Device Context} ( Windows-Seite ) verbinden. Hierzu mu"s zuerst die 
Win32-Funktion \emph{SetPixelFormat} und danach die Funktion \emph{wglCreateContext} mit
dem Device Context Handle Parameter aufgerufen werden. Ist dies erfolgreich,
gibt wglCreateContext einen Rendering Context Handle des Typs \emph{HGLRC} zur"uck.
OpenGL unter Windows kennt zwei Typen von Pixel-Modi:
Einen Modus, der die direkte Angabe von Pixelfarben erlaubt.
Und ein Modus, der die Auswahl der Farbe aus einer Palette unterst"utzt.

Es gibt spezielle Anforderungen durch OpenGL an ein Ausgabefenster.
Es mu"s mit den Fensterstilen \emph{WS\_CLIPSIBLINGS} und \emph{WS\_CLIPCHILDREN} initialisiert
sein um OpenGL Kompabilit"at zu gew"ahrleisten. 
Um die Perfomance der Applikation zu erh"ohen, sollte das Fenster einen
NULL-Hintergrund haben, da dieser sowieso durch die OpenGL-Bibliothek
gel"oscht wird.
Bevor ein Rendering Context verwendet werden kann, mu"s er mit der wglMakeCurrent
Funktion als der aktuelle Context bestimmt werden. Ist der Rendering Context
bereit um Kommandos entgegenzunehmen k"onnen weitere Initialisierung-Routinen
aufgerufen werden. Z.B. um den Frame Buffer zu l"oschen, Koordinatentransformationen
aufzusetzen, Lichtquellen zu konfigurieren oder andere Optionen zu setzen.

Ein solcher Initialisierungsschritt, der nicht ausgelassen werden darf,
ist der Aufruf der \emph{glViewport}-Funktion. Sie initialisiert oder modifiziert
die gr"o"se des\emph{ Rendering Viewports}. Typischerweise wird diese Funktion
ganz zu Anfang der Applikation aufgerufen und dann jedesmal wenn sie eine
\emph{WM\_SIZE}-Meldung erh"alt. Diese zeigt eine Gr"o"sen"anderung des Fensters an.


\chapter{Java Einf"uhrung\ddag}

Weitere englischsprachige Dokumentationen sind von Sun Microsystems erh"altlich\cite{JavaTuto}.

\section{Was ist Java?}
Sun Microsystems stellte Mitte 1995 sein Java Language Enviroment (kurz Java)
offiziell vor.

Java ist eine objektorientierte Programmiersprache und 
ist Plattformunabh"angig.

Java-Programme k"onnen entweder lokal auf einem Rechner ausgef"uhrt werden,
oder aber auch "uber das Internet gestartet werden.

\section{Grundlagen}

\subsection{Die Entwicklungsumgebung}
Sun Microsystems stellt f"ur die Plattformen Solaris(Sparc), Solaris(Intel) und Win32(Intel)\footnote{Windows 95 und Windows NT} die Java-Entwicklungsumgebung \emph{Java Development Kit (JDK)} kostenlos zur Verf"ugung. 
IBM stellt ihrerseits f"ur Windows 3.1(Intel), AIX(RS6000) und OS/2(Intel) die JDK zur Verf"ugung.

Die JDK enth"alt den Java-Compiler, die JVM und die standard Bibliothek, sowie
den Java-Debugger. Der Compiler als auch der Debugger sind Kommandozeilentools und enthalten keine GUI.
 
Mittlerweilen ist auch das \emph{Java Runtime Environment (JRE)} ver"offentlicht, welches lediglich die JVM und die standard Bibliothek enth"alt. Eigene Java-Applikationen k"onnen mit dem JRE geb"undelt und herausgegeben werden.

Der Java-Compiler generiert aus dem Java-Sourcecode den Java-Bytecode, wobei
dieser Java-Bytecode von der Java-Virtuellen-Maschiene (JVM) interpretiert wird.

Der Java-Sourcecode enth"alt die Klassendefinitionen.
In der Regel gibt es je Klassendefinition eine Quelldatei.
Der Dateiname der Quelldatei setzt sich zusammen aus dem Klassennamen und der Dateinamensendung \emph{.java}. Z.B. ist die Klassendefinition fuer \emph{Foo} in der Datei \emph{Foo.java} enthalten. Gross- und Kleinschreibung ist hierbei zu beachten.

Der Java-Compiler "ubersetzt die Quelldatei mit der Namensendung \emph{.java} in die Zieldatei f"ur den Java-Bytecode mit der Namensendung \emph{.class}.
Z.B. erzeugt der Aufruf von 

\begin{verbatim}
javac Foo.java
\end{verbatim}

die Datei \emph{Foo.class}.

Das compilierte Java-Programm, der Java-Bytecode, kann z.B. mittels

\begin{verbatim}
java Foo
\end{verbatim}

aufgerufen werden. Dieser Aufruf startet die JVM und beginnt das Programm
\emph{Foo} abzuarbeiten.

\subsection{Java-Virtuelle-Maschine (JVM)}
Die JVM simuliert eine komplette Betriebssystemumgebung
und ist in der Lage den Java-Bytecode abzuarbeiten.
Hiermit sind Java-Programme Plattform unabh"angig, da die JVM
auf verschiedenen Plattformen ein Einheitliches Java-System zur 
Verf"ugung stellen.

Mittels der JVM und den standard Java-Bibliotheken kann ein Java-Programm
\begin{itemize}
	\item auf das Netzwerk/Internet zugreifen
	\item eine grafische Oberfl"ache besitzen (AWT)
	\item auf native Funktionen zugreifen (JNI)
	\item auf Datenbanken zugreifen (jdbc)
	\item wiederverwendbare Komponente besitzen (Beans)
	\item gemeinsame standardisierte Objekte mit Java und 
	nicht-Java Applikationen austauschen (Corba/RMI)
	\item etc.
\end{itemize}

\subsection{Java-Interpreter und JIT-Compiler}
Der Java-Interpreter ist bestandteil der JVM. Der Java-Bytecode wird von dem Java-Interpreter abgearbeitet. Der JIT-Compiler (Just-In-Time-Compiler) ist eine Erweiterung des Java-Interpreters und "ubersetzt den Java-Bytecode blockweise in plattformspezifischen Machienencode.

\section{Java-Applets und Java-Applikationen}
Java-Programme lassen sich grob in zwei Kategorien teilen:
Java-Applikationen bezeichnen eigenst"andige Programme, welche innerhalb des ausf"uhrenden Rechners (Client) gestartet werden. Der Client fungiert hier gleichzeitig als Server. Java-Applikationen haben die M"oglichkeit auf das Dateisystem des Clients, bzw. Server zuzugreifen.
Z.B. kann die Java-Applikation Foo mittels 

\begin{verbatim}
java Foo
\end{verbatim}

aufgerufen werden. Hierf"ur werden lediglich die JVM und die entsprechenden \emph{.class}-Files ben"otigt.

Ein Java-Applet wird auf einem Server abgelegt. Das Applet wird auf dem Client "ubertragen und dort lokal ausgef"uhrt. Im Gegensatz zur Java-Applikation kann das Applet nicht auf Daten ausserhalb der JVM zugreifen. Somit kann ein Applet die Sicherheit des Systems nicht gef"ahrden.
Java-Applets werden innerhalb einer HTML\footnote{Hypertext Markup Language, Textformat f"ur World-Wide-Web(WWW) Seiten}-Seite aufgerufen.

Das Applet wird mittels des Internet, bzw. Intranet, auf den Client "ubetragen.
Hierf"ur kann ein Java-F"ahiger-WWW-Browser oder der \emph{appletviewer} des JDK benutzt werden. Z.B. kann das Java-Applet Foo eingebettet in der HTML-Seite \emph{Foo.html} mittels 
\begin{verbatim}
appletviewer http://www.server.de/Foo.html
\end{verbatim}
aufgerufen werden.
Mittlerweile haben die meisten Unternehmen Java lizensiert 
und in ihren Web-Browsern unterst"utzt.
Dadurch lassen sich Web-Seiten interaktiver gestalten.


\section{Objekt-Orientierte-Sprache}

\subsection{Was ist Objektorientierung?}
Java ist eine reine Objekt Orientierte Programmiersprache.
Im Gegensatz zu einer prozeduralen Programmiersprache, die sich durch einen
linearen Programmflu"s und einer klaren Trennung von Daten und 
Funktionen auszeichnet verkn"upft das objektorientierte Programmieren Daten und 
Methoden\footnote{Methoden ist der objektorientierte Name f"ur Funktionen} 
zu einem Objekt.
Objekte eines Typs lassen sich zu einer Objektklasse zusammenfassen.
Klassen sind Beschreibungen eines Datenmodells (Prototypen)
von denen sich beliebig viele Objekte generieren lassen.
In den Klassen werden sowohl die Daten als auch die damit verbundenen 
Operationen festgelegt.

\subsection{Klassen und Methoden}
In Java wird eine Klasse mittels des Schl"usselwortes \emph{class} (Analog zu
Klassen in C++) aufgerufen.

Der Zugriff auf Daten und Methoden k"onnen innerhalb einer Klasse 
f"ur andere Klassen mittels \emph{public}-deklaration erlaubt, bzw. 
mit \emph{private}- oder \emph{protected}-deklaration verboten werden.

Z.B. definiert die Klasse Feuerzeug Daten und Methoden f"ur Objekte des selben Typs.

\begin{verbatim}
public class Feuerzeug 
{ 
	/* Gasinhalt in Liter */
	private float gasInhalt; 

        /* 1 milliliter pro Zuendung */	
	private float finalize gasProZuendung = 0.001 

	/* Oeffentliche Methode fuer das Zuenden des Feuerzeugs.
           Beachte: Das Zuenden kostet Gas !
	*/
	public boolean zuende()
	{
		if(gasInhalt<gasProZuendung)
			return false;
		gasInhalt-=gasProZuendung;
		return true;
	}
}
\end{verbatim}


Eine Methode ist unmittelbar an ein Objekt bzw. die dazugeh"origen Daten
gebunden. Methoden k"onnen in Java auch "uberladen werden.
Z.B. 

\begin{verbatim}
public class Feuerzeug 
{ 
	...

	// Feuerzeug ganz aufladen
	void aufLaden()
	{
		...
	}

	// Feuerzeug mit definierter Menge 'vol' aufladen
	void aufLaden( float vol )
	{
		...
	}

}
\end{verbatim}


Durch Methoden werden bestimmte Funktionen eines Objektes beschrieben.
Sie werden in Java mittels des \emph{.} Operators aufgerufen.

Objekt.Methode (....)

Konstruktoren sind spezielle Methoden, die bei der Aktivierung von Objekten 
aufgerufen werden und deren Instanzierung und Initialisierung "ubernehmen.
Konstruktoren benutzen den gleichen Namen wie die verwendete Klasse.

Klasse Objekt=new KLasse (...)

erzeugt aus der Klasse ein Objekt, daf"ur reserviert es 
Speicher f"ur dieses Objekt
und initialisiert es mit den entsprechenden Werten ( evt. als Argumente "ubergeben).

\subsection{Vererbung}

Vererbung bezeichnet den Mechanismus des Ableitens einer Klasse aus einer 
"ubergeordneten Klasse (Einfachvererbung).
Die Daten und Methoden der "ubergeordneten Klasse, 
Oberklasse oder Superklasse genannt,
werden an die Subklasse weitergegeben (vererbt).
Die vererbten Methoden k"onnen dort auch "uberschrieben, bzw. "uberladen
werden.

Die Klassen der standard Bibliothek sind in der Regel alle von der superklasse
\emph{Object} abgeleitet. Die Klasse \emph{Object} definiert einige elementare Methoden (equals, string, getclass), welche von allen Subklassen benutzt, bzw. neu definiert werden k"onnen.

Im Gegensatz zu C++ k"onnen in Java keine Operatoren "uberladen, bzw. neu definiert werden. Als L"osung werden in Java standard Methoden verwendet, wie z.B. die Methode \emph{equals} der Superklasse \emph{Object}.

\subsection{Packages}

Ein wesentlicher Vorteil der objektorientierten Programmierung ist die
einfache Wiederverwendung bestehenden Programmcodes.
Klassenbibliotheken bieten Programmierern die m"oglichkeit 
bestehende Klassen in Ihrer Applikation einzubinden.
Es existieren eine Reihe von standard Klassenbibliotheken (Packages) f"ur 
unterschiedliche Zwecke wie z.B. Grafik, Netzwerkaktivit"aten und Ein-und
Ausgabefunktionen.
Packages werden mit dem \emph{import}-Befehl in den Programmcode eingebunden.
Die Zusammenfassung von mehreren Klassen in Packages erleichtert 
die Realisierung und Modularisierung gro"ser Systeme.
Packages dienen ebenfalls der einfachen Wiederverwendung.

\section{System-Managment}

\subsection{Garbage-Collector}
Ein in Java integrierter Garbage-Collector sorgt f"ur die automatische
Freigabe nicht mehr ben"otigter Speicherbereiche.
Dadurch werden potentielle Fehler vermieden, wie sie z.B. bei doppelter Speicherfreigabe in ANSI-C vorkommen k"onnen.
W"ahrend der Laufzeit eines Java-Programmes arbeitet im Hintergrund der 
Garbage Collector als Thread niedriger Priorit"at.

\subsection{Exceptions}
Mit den Exceptions besitzt Java einen Mechanismus zur strukturierten 
Behandlung von Fehlern, die w"ahrend der Programmausf"uhrung auftreten.
Tritt ein Laufzeitfehler (z.B. ein Array-Zugriff au"serhalb der Array-Grenzen), 
so wird ein eine entsprechende Exception generiert (\emph{throw}).
Diese Exception kann, bzw. mu"s innerhalb eines \emph{try-catch}-Blocks 
abgefangen werden (\emph{catch}). 

Das Grundprinzip des Exception-Mechanismus in Java kann wie folgt beschrieben
werden:
Wird eine Exception ausgel"ost, arbeitet die JVM den Stack soweit zur"uck,
bis sie eine entsprechende \emph{catch}-Anweisung findet. Wird keine geeignete gefunden, gibt die JVM eine Fehlermeldung aus.

Eine Exception kann an jeder Stelle der Aufrufkette (Stack) abgefangen werden.
Methoden, welche in der Deklaration angeben eine Exception ausl"osen zu k"onnen, m"ussen innerhalb eines \emph{try-catch} Blocks aufgerufen werden.

Das Behandeln von Exceptions mit der try-catch-Anweisung sieht etwa
folgenderma"sen aus :

\begin{verbatim}
try {
    Anweisung;  
     
    ...
    }
catch ( Exceptiontyp1 x ) {
    Anweisung;  

   ...
   }
catch ( Exceptiontyp2 x ) {
    Anweisung;  

   ...
   }
\end{verbatim}

\section{Grafisches User Interface (GUI)}

\subsection{Was ist eine GUI?}
AWT
Swing ..

\subsection{Abstract Window Toolkit (AWT)}

\subsection{Events}

Benutzer-Interaktion !

Bei der Programmierung unter einer grafischen Oberfl"ache erfolgt die
Kommunikation zwischen Betriebssystem und Anwendungsprogramm durch den 
Austausch von Nachrichten.
Die Anwendung wird dabei "uber alle Arten von Ereignissen und 
Zustands"anderungen vom Betriebssystem informiert.
Dazu z"ahlen z.B. Mausklicks, Bewegung des Mauszeigers, oder Ver"anderungen 
der Fenstergr"o"se oder dessen Lage.
Jedes dieser Ereignisse oder \emph{ Events} l"ost eine Nachricht aus, die an das 
aktive Fenster gesendet wird. Dadurch wird dort der Aufruf einer \emph{Callback}-Methode
ausgel"ost. Das Programm kann auf ein \emph{Even}t reagieren, indem es die zugeh"orige
Methode "uberlagert und mit der gew"unschten Funktionalit"at ausstattet.
Die unter Java m"oglichen Events lassen sich grob in folgende Klassen
unterteilen :
\begin{itemize}
\item Maus-Events
\item Tastatur-Events
\item Fenster-Events
\item Action-Events
\item Komponenten-Events
\end{itemize}

Z.B. wird ein Mausklick mit den Methoden \emph{mouseDown} und \emph{mouseUp} der 
Klasse \emph{Component} behendelt.
Wenn die Maustaste gedr"uckt wird, ruft das AWT die Methode \emph{mouseDown} auf,
l"asst man sie los wird die Methode \emph{mouseUp} aufgerufen.

public boolean mouseDown(Event e, int x, int y);
public boolean mouseUp(Event e. int x, int y);

Der erste Parameter beider Methoden ist eine Instanz Klasse \emph{Event}.
Die beiden anderen Parameter geben die Position in der Client-Area an,
an der die Maustaste gedr"uckt bzw. losgelassen wurde.
Ein Objekt der Klasse \emph{Event} repr"asentiert das Ereignis, durch das die
Nachricht ausgel"ost wurde.
Ein \emph{Event}-Objekt besitzt eine Reihe "offentlicher Instanzmerkmale
wie z.B.:


\begin{tabular}{|l|l|}
\hline
Element & Bedeutung \\
\hline
public int x; &	x-Koordinate des Mauszeigers bei Tastatur- \\
              & und Mausereignissen \\
\hline
public int y; &	y-Koordinate des Mauszeigers bei Tastatur- \\
              & und Mausereignissen \\
\hline
public long when; & Zeitpunkt des Ereignisses \\
\hline
public int key; & ASCII-Wert der gedr�ckten Taste bei \\
                & Tastaturereignissen bzw. Wert der Funktionstaste \\
\hline
public int modifiers; &	Eine Kombination der Konstanten \\
                      & \emph{Event.ALT\_MASK, Event.CTRL\_MASK,} \\
                      & \emph{Event.META\_MASK, Event.SHIFT\_MASK} \\
\hline
\end{tabular}


\section{Multithreading}

\subsection{Was ist Multithreading?}

Durch die Weiterentwicklungen im Bereich der Betriebssystemtechnologie
wurde das Konzept der \emph{Threads} in den letzten Jahren immer popul"arer.
Und durch bereitstellung der Basis von Library-Routinen auch konventionellen 
Programmiersprachen zur Verf"ugung gestellt.
Java hat \emph{Threads} direkt in die Sprache integriert und mit den erforderlichen
Hilfsmitteln als Konstrukt zur Nebenl"aufigkeit implementiert.
Ein \emph{Thread} ist ein eigenst"andiges Programmierfragment, das parallel
zu anderen \emph{Threads} laufen kann.
Der Laufzeit-Overhead zur Erzeugung und Verwaltung ist deutlich geringer
als bei gew"ohnlichen Prozessen und kann in den meisten Programmen 
vernachl"assigt werden.
\emph{Threads} sollen unter anderem die Implementierung grafischer Anwendungen
erleichtern, die durch Simulation komplexer Abl"aufe oft nebenl"aufig sind.
\emph{Threads} k"onnen auch dazu verwendet werden, die Bedienbarkeit von 
Dialoganwendungen zu verbessern, indem rechenintensive Anwendungen
im Hintergrund ablaufen.

\subsection{Programmierung von Threads}

\emph{Threads} werden in Java durch die Klasse\emph{ Thread} und das Interface \emph{Runnable}
implementiert. in beiden F"allen wird der \emph{Thread-Body}, also der
parallel auszuf"uhrende Code, in Form der "uberlagerten Methode \emph{run} zur
Verf"ugung gestellt. Die Kommunikation kann dann durch Zugriff auf die 
Instanz- oder Klassenvariablen oder durch Aufruf beliebiger Methoden,
die innerhalb von \emph{run} sichtbar sind, erfolgen.
Zur Synchronisation stellt Java das aus der Betriebssystemtheorie bekannte
Konzept des \emph{Monitors} zur Verf"ugung. Dises  erlaubtes, kritische Abschnitte
innerhalb korrekt geklammerter Programmfragmente und Methoden zu kapseln.
Somit wird der Zugriff auf gemeinsam benutzte Datenstrukturen koordiniert.
Dar"uber hinaus stellt Java Funktionen zur Verwaltung von \emph{Threads}.
Diese erlauben es, \emph{Threads} in Gruppen zusammenzufassen, zu priorisieren
und Informationen "uber Eigenschaften von \emph{Threads} zu gewinnen.
Das \emph{Schedulling} kann dabei wahlweise unterbrechend oder nichtunterbrechend
implementiert sein.
Die Sprachspezifikation legt dies nicht fest, aber in den meisten 
Java-Implementierungen wird dies von den M"oglichkeiten des darunter liegenden
Betriebssystems abh"angen.

Die Klasse \emph{Thread} ist Bestandteil des Packages \emph{java.lang} und steht damit
allen Anwendungen standartm"a"sig zur Verf"ugung.
\emph{Thread} stellt die Basismethoden zur Erzeugung, Kontrolle und zum Beenden von 
\emph{Threads} zur Verf"ugung. Um ein konkreten \emph{Thread} zu erzeugen, mu"s eine
eigene Klasse aus \emph{Thread} abgeleitet und die Methode \emph{run} "uberlagert werden.
Mit Hilfe des Aufrufs der Methode \emph{start} wird der \emph{Thread} gestartet und die weitere
Ausf"uhrung an die Methode \emph{run} "ubertragen.
\emph{start}wird nach dem starten des \emph{Threads} beendet, und der Aufrufer kann paralell
zum neu erzeugten \emph{ Thread} fortfahren.
Die "ubliche Vorgehensweise, einen \emph{Thread} zu beenden, besteht darin, die Methode \emph{stop}
der Klasse \emph{Thread} aufzurufen.
In diesem Fall wird der \emph{Thread} angehalten und aus der Liste der aktiven \emph{Threads}entfernt.
Mit Hilfe der Methoden \emph{suspend} und \emph{resume} ist es m"oglich, einen \emph{Thread} vor"ubergehend
 zu unterbrechen. Durch \emph{suspend} wird die Ausf"uhrug unterbrochen, und durch \emph{resume}
wird der \emph{Thread} (genauer gesagt: die Methode \emph{run})an der Stelle fortgesetzt, an der die
Unterbrechung erfolgte.


\section{Java Native Interface (JNI)}

\subsection{Was ist JNI?}
Java stellt mit den sogenannten \emph{nativen} Methoden die M"oglichkeit,
nicht in Java geschriebene Programmes oder Abl"aufe in die Laufzeitbibliothek
 mit einzubinden.
Gr"unde f�r die Implementierung von \emph{nativen} Methoden sind folgende:
\begin{itemize}
\item das man spezielle F"ahigkeiten der einzelnen Rechner oder der 
Betriebssysteme nutzen kann; die die Java-Bibliothek nicht bereitstellt.
Dazu geh"ort z.B. der Anschlu"s an neue Peripherieger"ate oder Steckkarten,
der Zugriff auf verschiedene Netztypen oder die Verwendung eines eindeutigen
Merkmals des vorhandenen Betriebssystems.
Solche F"ahigkeiten werden derzeit von der Java-Umgebung nicht bereitgestellt
und m"ussen daher \emph{au"serhalb} von Java in einer anderen Sprache (meist 
C oder eine mit C vertr"aglichen Sprache) implementiert werden.
\item Ein weiterer Grund ist die Performance des auszuf"uhrenden Programms.
Falls man den von Java zur Verf"ugung gestellten JIT-Compiler f"ur die 
Erh"ohung der Geschwindigkeit nicht einsetzen will, kann man f"ur den 
Geschwindigkeitsorientierten Teil ( z.B. kritische innere Schleifen ),
in C geschriebene \emph{native} Methoden einsetzen.
Die Java-Klassenbibliothek nutzt diese M"oglichkeit bei bestimmten 
kritischen Systemklassen selbst, um die Effiziens des Systems zu steigern.
\item Jedoch ist wohl der Hauptgrund f"ur den Einsatz von nativen Methoden in 
Java-Klassen der; da"s man auf diese Art und Weise schon existierende Programme
ohne sie erst umzuschreiben, relativ einfach in Java einbinden kann.
Diese M"oglichkeit haben wir in der Diplomarbeit auch genutzt.
Die 3D-OpenGL-Animationen sind in C geschrieben und wurden sozusagen
in Java eingebettet, doch dazu sp"ater mehr.
\end{itemize}

\subsection{Grundlagen der JNI}

Dynamische Bibliothek
	Unix: lib<Name>.so, LD\_LIBRARY\_PATH
	WIN: <Name>.dll, SYSTEMVERZEICHNIS

\subsection{Vorgehensweise}

Um ein Interface zwischen Java und dem C-Code zu erstellen, sind im wesenlichen
f"unf Schritte erforderlich :
\begin{enumerate}
	\item Erstellen der Java-JNI Funktionsdeklaration
	\item Generierung der C-Header Datei
	\item Erstellen der C-JNI Functionen
	\item Laden der JNI-Bibliothek in Java
	\item Linken der Dynamischen Bibliothek
\end{enumerate}

\subsection{Erstellen der Java-JNI Funktionsdeklaration}
Den Java-Code mit der deklarierten \emph{nativen} Methode zu schreiben und zu 
"Ubersetzen.

Um \emph{native} Methoden in Java-Klassen zu benutzen, mu"s man in der Deklaration
der Methode lediglich das schl"usselwort \emph{native} einf"ugen.

wie z.B. 
\begin{verbatim}
public nativ void Methodenname();
\end{verbatim}

Und die Methode System.loadLibrary() in die Klasse mit einbinden.
Das Schl"usselwort \emph{native} zeigt dem javac-Compiler und dem java-Interpreter
das sie nach einem Methodenrumpf in einer dynamisch-ladbaren Bibliothek
Ausschau halten m"ussen.
Um die gew"unschte Bibliothek letzendlich wirklich aufzurufen, wird sie mit der
Methode \emph{System.loadLibrary()} aus dem System-Package von Java eingelesen.
Java durchsucht alle Pfade die in dem Betriebssystem-Umgebungsvariable
angegeben wurden.
Alternativ steht die Methode\emph{ Runtime.getRuntime().loadLibrary()} und die 
Methode \emph{System.load()} zur Verf"ugung.
Letztrige liest eine Klassen-Bibliothek anhand ihres vollst"andigen 
Pfadnamens ein. So k"onnen auch Bibliotheken au"serhalb des Suchpfades 
benutzt werden.
Die "Ubersetzung des Java-Codes kann wie "ublich ausgef"uhrt werden.

\subsection{Generierung der C-Header Datei}
Der n"achste Schritt ist es, die "ubersetzte Datei \emph{Datei.class} dazu zu benutzen, um die
entsprechende Header-Datei \emph{Datei.h} zu erzeugen.
Dazu wird das \emph{javah}-Tool aus der JDK-Distribution eingesetzt.
Per Default wird die neue h-Datei im gleichen Verzeichnis angelegt. Mit der 
Option -d kann ein anderes Verzeichnis angegeben werden. 
Dies ist zu raten da noch diverse andere Dateien erzeugt werden m"ussen und
schnell die "Ubersicht verloren gehen kann.
Der Aufruf 

\begin{verbatim}
javah -jni Datei
\end{verbatim}

erzeugt das C-Headerfile \emph{Datei.h}. 

In der Header-Datei wird durch die typedef-struct-Anweisung der \emph{nativen}
C-Routine mitgeteilt, wie die Daten in der Java-Klasse angeordnet sind. Die einzelnen Variablen
in der Struktur k"onnen benutzt werden, um die Klassen
und Instanz-Variablen von Java zu benutzen.
Weiterhin wird in dem h.File ein Prototype angegeben, um die Methode aus dem 
objektorientierten Namensraum der Java Klasse in den C-Namensraum zu 
"uberf"uhren.
Der Name einer Funktion, der eine \emph{native} Methode implementiert, ergibt sich 
dabei immer aus dem Packet-Namen, dem Klassen-Namen und dem Namen der nativen
Methode von Java, getrennt durch einen Unterstrich.

\subsection{Erstellen der C-JNI Functionen}
Nun m"ussen die nativen Methoden implementiert werden.
"Ublicherweise werden die Implementationsdateien mit dem Klassennamen und einer
eindeutigen Endung z.B. \emph{Datei.c} versehen.
Die Implementationsdatei mu"s die Datei \emph{jni.h} mittels \#include-Anweisung einbinden. \emph{jni.h} ist im \emph{include} Verzeichniss des JDK enthalten.
Ebenfalls muss die mittels \emph{javah -jni} erzeugte Headerdatei eingebunden werden. 
Der Funktionskopf in der Implementationsdatei muss den 
generierten Prototypen aus der Headerdatei entsprechen.

Diese Fehlerart macht sich noch nicht beim Linken sondern erst zur Laufzeit 
des Programms bemerkbar und ist daher m"uhsam zu beheben.

\subsection{Erstellung der Dynamischen Bibliothek}

Nachdem nun alle ben"otigten Dateien vorhanden sind, mu"s die 
Implemtierungsdatei nur noch "ubersetzt und mit der Java-Bibliothek  
zu einer dynamischen Bibliothek zusammen gelinkt werden.
  
\subsection{Datenaustausch zwischen Java und C}

Jedem elementaren Typ in Java wird ein Typ in der Prototype-Funktion
und damit auch in C zugeordnet
Ein char in Java ist nach dem UNICODE kodiert, im Gegensatz zu ASCII in C.
Dem Java-Typ String kann nur eine Struktur zugeordnet werden. 
Funktionen zur Umwandlung von UNICODE und ASCII sowie die Definition
String-Strukturen finden sich in der Header-Datei wieder javaString.h .
 
Variablen nicht als Parameter einer Methode sondern als Klassenvariable
anzulegen ist objektorientiert und erm"oglicht einen einfachen Zugriff
von C aus. Dieser Zugriff kann als 'Call by Reference' bezeichnet werden.
"Ubergibt man die Variablen als Parameter ist der Zugriff ein
Call by Value.
Alle Klassenvariablen werden im .h-File zu einem \emph{struct} zusammengefa"st.
Auf diesen erh"alt die aufgerufene C-Funktion als Parameter ein
Handle-Pointer.
Das in \emph{StubPreamble.h} definierte, \emph{unhand()}-Makro erm"oglicht den Zugriff
auf die einzelnen Klassenvariablen. Das \emph{unhand()}-Makro "ubernimmt einen
Pointer auf das Handle einer Klasse und gibt einen Pointer auf die im 
.h-File erzeugte Klassenstruktur zur"uck.
"Uber den R"uckgabewert des Makros lassen sich die Instance-Variablen
der Java-Klasse direkt auswerten und ver"andern.

\subsubsection{Strings}
Wie schon erw"ahnt bilden \emph{Strings} in Java eine eigene Klasse.
M"ochte man einen \emph{String} von Java nach C oder umgekehrt "ubergeben,
mu"s man \emph{javaString.h} im C-Implementationsfile mit einbinden.
In \emph{javaString.h} finden sich die Typdefintionen und Funktionen um \emph{Strings}
von Java nach C und umgekehrt zu transformieren.
So gibt es noch weitere Methoden um \emph{Strings} zu bearbeiten, wie z.B.
\emph{MoveString\_GetString(), CString()} oder \emph{makeJavaString()}.
Java-\emph{Strings} bilden eine eigene Klasse, somit wird bei der Konvertierung nicht 
nur ein elementarer Typ sondern eine ganze Klasse an die \emph{native} Methode
durchgereicht. Dieses Konzept l"a"st sich auch auf andere Klassen erweitern.

\subsubsection{Felder}
\emph{Felder} werden von Java nach C "ubergeben indem vom \emph{javah}-Tool ein 
Funktionsparameter von Typ \emph{struct HArrayOf<Objekt> *} erzeugt wird.
Dieser Handle enth"alt ein Element \emph{body} mit dem auf die Feldelemente 
zugegriffen werden kann.
Der Zugriff auf das Feld erfolgt mit dem \emph{unhand()}-Makro und dem
\emph{body}-Element.

\subsubsection{Speichermanagment}
in Java wird die Speicherverwaltung automatisch vom \emph{Garbage-Collector} 
erledigt indem er die Methode \emph{dispose()} aus der Java-Klasse ausf"uhrt.
Wird in einer C-Funktion einer \emph{nativen} Methode Speicher angelegt,
so h"angt die Vorgehensweise des Programmierers beim Freigeben,
von der Art des Speichers ab.
Bei normalen Speicherbereichen f"ur die interne C-Anwendung mu"s der 
Programmierer sich explizit um die Freigabe k"ummern. 
Da der Java-Interpreter von diesem Speicher nichts wei"s,
m"ussen eigene \emph{native} Methoden implementiert werden um ihn freizugeben.
In C erzeugte Speicherbereiche von Java-Objekten werden nicht unbedingt
vom \emph{Garbage-Collector} gefunden. Diese Java-Objekte, also Instanzen 
einer bestimmten Java-Klasse, k"onnen auch aus Java heraus freigegeben 
werden. Eine einfache Variante der Freigabe ist der explizite Aufruf
der Methode \emph{dispose()} wenn die Instanz nicht mehr ben"otigt wird.
%
%\begin{thebibliography}{99}
%	\bibitem{1} Mary Campione and Kathy Walrath; The Java Tutorial; Sun Microsystems; http://www.javasoft.com/docs/books/tutorial/index.html
%	\bibitem{2} Ron Fosner; OpenGL Programming for Windows 95 and Windows NT; Addison Wesley Second Printing April 1997; http://www.aw.com/devpress
%	\bibitem{3} Woo, Mason; OpenGL programming guide 2nd ed., Addison Wesley Developer Press April 1997, ISBN 0-0201-46138-2; http://www.aw.com/devpress
%\end{thebibliography}
%
%\end{document}
% 
